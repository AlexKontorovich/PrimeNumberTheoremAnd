
\begin{theorem}[FKS2 Theorem 6]\label{FKS2_theorem_6}  Let $x_0 > 0$ be chosen such that $\pi(x_0)$ and $\theta(x_0)$ are computable, and let $x_1 \geq \max(x_0, 14)$. Let $\{b_i\}_{i=1}^N$ be a finite partition of $[\log x_0, \log x_1]$, with $b_1 = \log x_0$ and $b_N = \log x_1$, and suppose that $\varepsilon_{\theta,\mathrm{num}}$ gives computable admissible numerical bounds for $x = \exp(b_i)$, for each $i=1,\dots,N$.  For $x_1 \leq x_2 \leq x_1 \log x_1$, we define
    $$ \mu_{num}(x_0,x_1,x_2) = \frac{x_0 \log x_1}{\varepsilon_{\theta,num}(x_0) x_1 \log x_0} \left|\frac{\pi(x_0) - \Li(x_0)}{x_0/\log x_0} - \frac{\theta(x_0) - x_0}{x_0}\right|$$

$$ + \frac{\log x_1}{\varepsilon_{theta,num}(x_1) x_1} \sum_{i=1}^{N-1} \varepsilon_{\theta,num}(\exp(b_i)) \left( Li(e^{b_{i+1}}) - Li(e^{b_i}) + \frac{e^{b_i}}{b_i} - \frac{e^{b_{i+1}}}{b_{i+1}}$$
$$ + \frac{\log x_2}{x_2} \left( Li(x_2) - \frac{x_2}{\log x_2} - Li(x_1) + \frac{x_1}{\log x_1} \right)$$
and for $x_2 > x_1 \log x_1$, including the case $x_2 = \infty$, we define
$$ \mu_{num}(x_0,x_1,x_2) = \frac{x_0 \log x_1}{\varepsilon_{\theta,num}(x_1) x_1 \log x_0} \left|\frac{\pi(x_0) - \Li(x_0)}{x_0/\log x_0} - \frac{\theta(x_0) - x_0}{x_0}\right|$$
$$ + \frac{\log x_1}{\varepsilon_{\theta,num}(x_1) x_1} \sum_{i=1}^{N-1} \varepsilon_{\theta,num}(\exp(b_i)) \left( Li(e^{b_{i+1}}) - Li(e^{b_i}) + \frac{e^{b_i}}{b_i} - \frac{e^{b_{i+1}}}{b_{i+1}} \right)$$
$$ + \frac{1}{\log x_1 + \log\log x_1 - 1}.$$
Then, for all $x_1 \leq x \leq x_2$ we have
$$ E_\pi(x) \leq \varepsilon_{\pi,num}(x_1,x_2) := \varepsilon_{\theta,num}(x_1)(1 + \mu_{num}(x_0,x_1,x_2)).$$
\end{theorem}

\begin{proposition}[FKS2 Remark 7]\label{FKS2_remark_7} If
    $$ \frac{d}{dx} \frac{\log x}{x} \left( Li(x) - \frac{x}{\log x} - Li(x_1) + \frac{x_1}{\log x_1} \right)|_{x_2} \geq 0 $$
    then $\mu_{num,1}(x_0,x_1,x_2) < \mu_{num,2}(x_0,x_1)$.
\end{proposition}

\begin{corollary}[FKS2 Corollary 8]\label{FKS2_corollary_8} Let $\{b'_i\}_{i=1}^M$ be a set of finite subdivisions of $[\log(x_1),\infty)$, with $b'_1 = \log(x_1)$ and $b'_M = \infty$. Define
    $$ \varepsilon_{\pi, num}(x_1) := \max_{1 \leq i \leq M-1}(\exp(b'_i), \exp(b'_{i+1})).$$
    Then $E_\pi(x) \leq \varepsilon_{\pi,num}(x_1)$ for all $x \geq x_1$.
\end{corollary}

\begin{lemma}[FKS2 equation (17)]\label{FKS2_eq_17} For any $2 \leq x_0 < x$ one has
    $$ (\pi(x) - Li(x)) - (\pi(x_0) - Li(x_0)) = \frac{\theta(x) - x}{\log x} - \frac{\theta(x_0) - x_0}{\log x_0} + \int_{x_0}^x \frac{\theta(t) - t}{t \log^2 t} dt.$$
\end{lemma}

\begin{lemma}[FKS2 Lemma 10a]\label{FKS2_lemma_10a} If $a>0$, $c>0$ and $b < -c^2/16a$, then $g(a,b,c,x)$ decreases with $x$.
\end{lemma}

\begin{lemma}[FKS2 Lemma 10b]\label{FKS2_lemma_10b} For any $a>0$, $c>0$ and $b \geq -c^2/16a$, $g(a,b,c,x)$ decreases with $x$ for $x > \exp((\frac{c}{4a} + \frac{1}{2a} \sqrt{\frac{c^2}{4} + 4ab})^2)$.
\end{lemma}

\begin{lemma}[FKS2 Lemma 10c]\label{FKS2_lemma_10c} If $c>0$, $g(0,b,c,x)$ decreases with $x$ for $\sqrt{\log x} > -2b/c$.
\end{lemma}

\begin{corollary}[FKS2 Corollary 11]\label{FKS2_corollary_11} If $B \geq 1 + C^2 / 16R$ then $g(1,1-B,C/\sqrt{R},x)$ is decreasing in $x$.
\end{corollary}

\begin{lemma}[FKS2 remark after Corollary 11]\label{FKS2_remark_after_corollary_11} The Dawson function has a single maximum at $x \approx 0.942$, after which the function is decreasing.
\end{lemma}

\begin{lemma}[FKS2 Lemma 12]\label{FKS2_lemma_12} Suppose that $E_\theta$ satisfies an admissible classical bound with parameters $A,B,C,R,x_0$. Then, for all $x \geq x_0$,
    $$ \int_{x_0}^x |\frac{E_\theta(t)}{\log^2 t} dt| \leq \frac{2A}{R^B} x m(x_0,x) \exp(-C \sqrt{\frac{\log x}{R}}) D_+( \sqrt{\log x} - \frac{C}{2\sqrt{R} )$$
    where
    $$ m(x_0,x) = \max ( (\log x_0)^{(2B-3)/2}, (\log x)^{(2B-3)/2} ). $$
\end{lemma}

\begin{corollary}[BKLNW Corollary 14.1]\label{BKLNW_corollary_14_1} Suppose that $A_\psi,B,C,R,x_0$ give an admissible bound for $E_\psi$.  If $B > C^2/8R$, then $A_\theta, B, C, R, x_0$ give an admissible bound for $E_\theta$, where
    $$ A_\theta = A_\psi (1 + \nu_{asymp}(x_0))$$
with
$$ \nu_{asymp}(x_0) = \frac{1}{A_\psi} (\frac{R}{\log x_0})^B \exp(C \sqrt{\frac{\log x_0}{R}}) (a_1 (\log x_0) x_0^{-1/2} + a_2 (\log x_0) x_0^{-2/3}).$$
\end{corollary}



\begin{corollary}[FKS2 Corollary 14]\label{FKS2_corollary_14} We have an admissible bound for $E_\theta$ with $A = 121.0961$, $B=3/2$, $C=2$, $R = 5.5666305$, $x_0=2$.
\end{corollary}


\begin{theorem}[FKS Theorem 1.2b]\label{FKS_theorem_1_2b} If $\log x_0 \geq 1000$ then we have an admissible bound for $E_\psi$ with the indicated choice of $A(x_0)$, $B = 3/2$, $C = 2$, and $R = 5.5666305$.
\end{theorem}

\begin{theorem}[FKS2 Remark 15]\label{FKS2_remark_15} If $\log x_0 \geq 1000$ then we have an admissible bound for $E_\theta$ with the indicated choice of $A(x_0)$, $B = 3/2$, $C = 2$, and $R = 5.5666305$.
\end{theorem}

\begin{proposition}[FKS2 Proposition 17]\label{FKS2_proposition_17} Let $x > x_0 > 2$.  IF $E_\psi(x) \leq \varepsilon_{\psi,num}(x_0)$, then
    $$ - \varepsilon_{\theta,num}(x_0) \leq \frac{\theta(x)-x}{x} \leq \varepsilon_{\psi,num}(x_0) \leq \varepsilon_{\theta,num}(x_0)$$
  where
  $$ \varepsilon_{\theta,num}(x_0) = \varepsilon_{\psi,num}(x_0) + 1.00000002(x_0^{-1/2}+x_0^{-2/3}+x_0^{-4/5}) + 0.94 (x_0^{-3/4} + x_0^{-5/6} + x_0^{-9/10})$$
\end{proposition}

\begin{lemma}[FKS2 Lemma 19]\label{FKS2_lemma_19} Let $x_1 > x_0 \geq 2$, $N \in \N$, and let $(b_i)_{i=1}^N$ be a finite partition of $[x_0,x_1]$.  Then
    $$ |\int_{x_0}^{x_1} \frac{\theta(t)-t}{t \log^2 t}\ dt| \leq \sum_{i=1}^{N-1} \vareps_{\theta,num}(e^{b_i}) (Li(e^{b_{i+1}}) - Li(e^{b_i}) + \frac{e^{b_i}}{b_i} - \frac{e^{b_{i+1}}}{b_{i+1}}).$$
\end{lemma}

\begin{lemma}[FKS2 Lemma 20]\label{FKS2_lemma_20} Assume $x \geq 6.58$. Then $Li(x) - \frac{x}{\log x}$ is strictly increasing and $Li(x) - \frac{x}{\log x} > \frac{x-6.58}{\log^2 x} > 0$.
\end{lemma}

\begin{corollary}[FKS2 Corollary 21]\label{FKS2_corollary_21} Let $B \geq \max(\frac{3}{2}, 1 + \frac{C^2}{16R})$.  Let $x_0, x_1 > 0$ with $x_1 \geq \max(x_0, \exp( (1 + \frac{C}{2\sqrt{R}})^2))$. If $E_\psi$ satisfies an admissible classical bound with parameters $A_\psi,B,C,R,x_0$, then $E_\pi$ satisfies an admissible classical bound with $A_\pi, B, C, R, x_1$, where
    $$ A_\pi = (1 + \nu_{asymp}(x_0)) (1 + \mu_{asymp}(x_0, x_1)) A_\psi$$
    for all $x \geq x_0$, where
    $$ |E_\theta(x)| \leq \varepsilon_{\theta,asymp}(x) := A (1 + \mu_{asymp}(x_0,x)) \exp(-C \sqrt{\frac{\log x}{R}})$$
    where
    $$ \nu_{asymp}(x_0) = \frac{1}{A_\psi} (\frac{R}{\log x_0})^B \exp(C \sqrt{\frac{\log x_0}{R}}) (a_1 (\log x_0) x_0^{-1/2} + a_2 (\log x_0) x_0^{-2/3})$$
and
    $$ \mu_{asymp}(x_0,x_1) = \frac{x_0 \log x_1}{\eps_{\theta,asymp}(x_1)x_1 \log x_0} |E_\pi(x_0) - E_\theta(x_0)| + \frac{2 D_+(\sqrt{\log x} - \frac{C}{2\sqrt{R}})}{\sqrt{\log x_1}}.$$
\end{corollary}

\begin{corollary}[FKS2 Corollary 23]\label{FKS2_corollary_23} $A_\pi, B, C, x_0$ as in Table 6 give an admissible asymptotic bound for $E_\pi$ with $R = 5.5666305$.
\end{corollary}


\begin{corollary}[FKS2 Corollary 24]\label{FKS2_corollary_24} We have the bounds $E_\pi(x) \leq B(x)$, where $B(x)$ is given by Table 7.
\end{corollary}