
The approach here is completely standard. We follow the use of
$\mathcal{M}(\widetilde{1_{\epsilon}})$ as in [Kontorovich 2015].


\begin{definition}\label{ChebyshevPsi}\lean{ChebyshevPsi}\leanok
The (second) Chebyshev Psi function is defined as
$$
\psi(x) := \sum_{n \le x} \Lambda(n),
$$
where $\Lambda(n)$ is the von Mangoldt function.
\end{definition}


It has already been established that zeta doesn't vanish on the 1 line, and has a pole at $s=1$
of order 1.
We also have the following.
\begin{theorem}[LogDerivativeDirichlet]\label{LogDerivativeDirichlet}\lean{LogDerivativeDirichlet}\leanok
We have that, for $\Re(s)>1$,
$$
-\frac{\zeta'(s)}{\zeta(s)} = \sum_{n=1}^\infty \frac{\Lambda(n)}{n^s}.
$$
\end{theorem}


\begin{proof}\leanok
Already in Mathlib.
\end{proof}


The main object of study is the following inverse Mellin-type transform, which will turn out to
be a smoothed Chebyshev function.

\begin{definition}[SmoothedChebyshev]\label{SmoothedChebyshev}\lean{SmoothedChebyshev}\leanok
Fix $\epsilon>0$, and a bumpfunction supported in $[1/2,2]$. Then we define the smoothed
Chebyshev function $\psi_{\epsilon}$ from $\mathbb{R}_{>0}$ to $\mathbb{C}$ by
$$\psi_{\epsilon}(X) = \frac{1}{2\pi i}\int_{(\sigma)}\frac{-\zeta'(s)}{\zeta(s)}
\mathcal{M}(\widetilde{1_{\epsilon}})(s)
X^{s}ds,$$
where we'll take $\sigma = 1 + 1 / \log X$.
\end{definition}


\begin{lemma}[SmoothedChebyshevIntegrand_conj]\label{SmoothedChebyshevIntegrand_conj}\lean{SmoothedChebyshevIntegrand_conj}\leanok
The smoothed Chebyshev integrand satisfies the conjugation symmetry
$$
\psi_{\epsilon}(X)(\overline{s}) = \overline{\psi_{\epsilon}(X)(s)}
$$
for all $s \in \mathbb{C}$, $X > 0$, and $\epsilon > 0$.
\end{lemma}


\begin{proof}\leanok
\ uses{deriv_riemannZeta_conj, riemannZeta_conj}
We expand the definition of the smoothed Chebyshev integrand and compute, using the corresponding
conjugation symmetries of the Riemann zeta function and its derivative.
\end{proof}


\begin{lemma}[SmoothedChebyshevDirichlet_aux_integrable]\label{SmoothedChebyshevDirichlet_aux_integrable}\lean{SmoothedChebyshevDirichlet_aux_integrable}\leanok
Fix a nonnegative, continuously differentiable function $F$ on $\mathbb{R}$ with support in $[1/2,2]$, and total mass one, $\int_{(0,\infty)} F(x)/x dx = 1$. Then for any $\epsilon>0$, and $\sigma\in (1, 2]$, the function
$$
x \mapsto\mathcal{M}(\widetilde{1_{\epsilon}})(\sigma + ix)
$$
is integrable on $\mathbb{R}$.
\end{lemma}


\begin{proof}\leanok
\uses{MellinOfSmooth1b}
By Lemma \ref{MellinOfSmooth1b} the integrand is $O(1/t^2)$ as $t\rightarrow \infty$ and hence the function is integrable.
\end{proof}


\begin{lemma}[SmoothedChebyshevDirichlet_aux_tsum_integral]\label{SmoothedChebyshevDirichlet_aux_tsum_integral}
\lean{SmoothedChebyshevDirichlet_aux_tsum_integral}\leanok
Fix a nonnegative, continuously differentiable function $F$ on $\mathbb{R}$ with support in
$[1/2,2]$, and total mass one, $\int_{(0,\infty)} F(x)/x dx = 1$. Then for any $\epsilon>0$ and $\sigma\in(1,2]$, the
function
$x \mapsto \sum_{n=1}^\infty \frac{\Lambda(n)}{n^{\sigma+it}}
\mathcal{M}(\widetilde{1_{\epsilon}})(\sigma+it) x^{\sigma+it}$ is equal to
$\sum_{n=1}^\infty \int_{(0,\infty)} \frac{\Lambda(n)}{n^{\sigma+it}}
\mathcal{M}(\widetilde{1_{\epsilon}})(\sigma+it) x^{\sigma+it}$.
\end{lemma}


\begin{proof}\leanok
\uses{Smooth1Properties_above, SmoothedChebyshevDirichlet_aux_integrable}
Interchange of summation and integration.
\end{proof}


Inserting the Dirichlet series expansion of the log derivative of zeta, we get the following.
\begin{theorem}[SmoothedChebyshevDirichlet]\label{SmoothedChebyshevDirichlet}
\lean{SmoothedChebyshevDirichlet}\leanok
We have that
$$\psi_{\epsilon}(X) = \sum_{n=1}^\infty \Lambda(n)\widetilde{1_{\epsilon}}(n/X).$$
\end{theorem}


\begin{proof}\leanok
\uses{SmoothedChebyshev, LogDerivativeDirichlet, Smooth1LeOne, MellinOfSmooth1b,
SmoothedChebyshevDirichlet_aux_integrable,
Smooth1ContinuousAt, SmoothedChebyshevDirichlet_aux_tsum_integral}
We have that
$$\psi_{\epsilon}(X) = \frac{1}{2\pi i}\int_{(2)}\sum_{n=1}^\infty \frac{\Lambda(n)}{n^s}
\mathcal{M}(\widetilde{1_{\epsilon}})(s)
X^{s}ds.$$
We have enough decay (thanks to quadratic decay of $\mathcal{M}(\widetilde{1_{\epsilon}})$) to
justify the interchange of summation and integration. We then get
$$\psi_{\epsilon}(X) =
\sum_{n=1}^\infty \Lambda(n)\frac{1}{2\pi i}\int_{(2)}
\mathcal{M}(\widetilde{1_{\epsilon}})(s)
(n/X)^{-s}
ds
$$
and apply the Mellin inversion formula.
\end{proof}


The smoothed Chebyshev function is close to the actual Chebyshev function.
\begin{theorem}[SmoothedChebyshevClose]\label{SmoothedChebyshevClose}\lean{SmoothedChebyshevClose}\leanok
We have that
$$\psi_{\epsilon}(X) = \psi(X) + O(\epsilon X \log X).$$
\end{theorem}


\begin{proof}\leanok
\uses{SmoothedChebyshevDirichlet, Smooth1Properties_above,
Smooth1Properties_below,
Smooth1Nonneg,
Smooth1LeOne,
ChebyshevPsi}
Take the difference. By Lemma \ref{Smooth1Properties_above} and \ref{Smooth1Properties_below},
the sums agree except when $1-c \epsilon \leq n/X \leq 1+c \epsilon$. This is an interval of
length $\ll \epsilon X$, and the summands are bounded by $\Lambda(n) \ll \log X$.

%[No longer relevant, as we will do better than any power of log savings...: This is not enough,
%as it loses a log! (Which is fine if our target is the strong PNT, with
%exp-root-log savings, but not here with the ``softer'' approach.) So we will need something like
%the Selberg sieve (already in Mathlib? Or close?) to conclude that the number of primes in this
%interval is $\ll \epsilon X / \log X + 1$.
%(The number of prime powers is $\ll X^{1/2}$.)
%And multiplying that by $\Lambda (n) \ll \log X$ gives the desired bound.]
\end{proof}


Returning to the definition of $\psi_{\epsilon}$, fix a large $T$ to be chosen later, and set
$\sigma_0 = 1 + 1 / log X$,
$\sigma_1 = 1- A/ \log T^9$, and
$\sigma_2<\sigma_1$ a constant.
Pull
contours (via rectangles!) to go
from $\sigma_0-i\infty$ up to $\sigma_0-iT$, then over to $\sigma_1-iT$,
up to $\sigma_1-3i$, over to $\sigma_2-3i$, up to $\sigma_2+3i$, back over to $\sigma_1+3i$, up to $\sigma_1+iT$, over to $\sigma_0+iT$, and finally up to $\sigma_0+i\infty$.

\begin{verbatim}
                    |
                    | I₉
              +-----+
              |  I₈
              |
           I₇ |
              |
              |
  +-----------+
  |       I₆
I₅|
--σ₂----------σ₁--1-σ₀----
  |
  |       I₄
  +-----------+
              |
              |
            I₃|
              |
              |  I₂
              +-----+
                    | I₁
                    |
\end{verbatim}

In the process, we will pick up the residue at $s=1$.
We will do this in several stages. Here the interval integrals are defined as follows:


\begin{definition}[I₁]\label{I1}\lean{I₁}\leanok
$$
I_1(\nu, \epsilon, X, T) := \frac{1}{2\pi i} \int_{-\infty}^{-T}
\left(
\frac{-\zeta'}\zeta(\sigma_0 + t i)
\right)
 \mathcal M(\widetilde 1_\epsilon)(\sigma_0 + t i)
X^{\sigma_0 + t i}
\ i \ dt
$$
\end{definition}


\begin{definition}[I₂]\label{I2}\lean{I₂}\leanok
$$
I_2(\nu, \epsilon, X, T, \sigma_1) := \frac{1}{2\pi i} \int_{\sigma_1}^{\sigma_0}
\left(
\frac{-\zeta'}\zeta(\sigma - i T)
\right)
  \mathcal M(\widetilde 1_\epsilon)(\sigma - i T)
X^{\sigma - i T} \ d\sigma
$$
\end{definition}


\begin{definition}[I₃₇]\label{I37}\lean{I₃₇}\leanok
$$
I_{37}(\nu, \epsilon, X, T, \sigma_1) := \frac{1}{2\pi i} \int_{-T}^{T}
\left(
\frac{-\zeta'}\zeta(\sigma_1 + t i)
\right)
  \mathcal M(\widetilde 1_\epsilon)(\sigma_1 + t i)
X^{\sigma_1 + t i} \ i \ dt
$$
\end{definition}


\begin{definition}[I₈]\label{I8}\lean{I₈}\leanok
$$
I_8(\nu, \epsilon, X, T, \sigma_1) := \frac{1}{2\pi i} \int_{\sigma_1}^{\sigma_0}
\left(
\frac{-\zeta'}\zeta(\sigma + T i)
\right)
  \mathcal M(\widetilde 1_\epsilon)(\sigma + T i)
X^{\sigma + T i} \ d\sigma
$$
\end{definition}


\begin{definition}[I₉]\label{I9}\lean{I₉}\leanok
$$
I_9(\nu, \epsilon, X, T) := \frac{1}{2\pi i} \int_{T}^{\infty}
\left(
\frac{-\zeta'}\zeta(\sigma_0 + t i)
\right)
  \mathcal M(\widetilde 1_\epsilon)(\sigma_0 + t i)
X^{\sigma_0 + t i} \ i \ dt
$$
\end{definition}


\begin{definition}[I₃]\label{I3}\lean{I₃}\leanok
$$
I_3(\nu, \epsilon, X, T, \sigma_1) := \frac{1}{2\pi i} \int_{-T}^{-3}
\left(
\frac{-\zeta'}\zeta(\sigma_1 + t i)
\right)
  \mathcal M(\widetilde 1_\epsilon)(\sigma_1 + t i)
X^{\sigma_1 + t i} \ i \ dt
$$
\end{definition}

\begin{definition}[I₇]\label{I7}\lean{I₇}\leanok
$$
I_7(\nu, \epsilon, X, T, \sigma_1) := \frac{1}{2\pi i} \int_{3}^{T}
\left(
\frac{-\zeta'}\zeta(\sigma_1 + t i)
\right)
  \mathcal M(\widetilde 1_\epsilon)(\sigma_1 + t i)
X^{\sigma_1 + t i} \ i \ dt
$$
\end{definition}


\begin{definition}[I₄]\label{I4}\lean{I₄}\leanok
$$
I_4(\nu, \epsilon, X, \sigma_1, \sigma_2) := \frac{1}{2\pi i} \int_{\sigma_2}^{\sigma_1}
\left(
\frac{-\zeta'}\zeta(\sigma - 3 i)
\right)
  \mathcal M(\widetilde 1_\epsilon)(\sigma - 3 i)
X^{\sigma - 3 i} \ d\sigma
$$
\end{definition}


\begin{definition}[I₆]\label{I6}\lean{I₆}\leanok
$$
I_6(\nu, \epsilon, X, \sigma_1, \sigma_2) := \frac{1}{2\pi i} \int_{\sigma_2}^{\sigma_1}
\left(
\frac{-\zeta'}\zeta(\sigma + 3 i)
\right)
  \mathcal M(\widetilde 1_\epsilon)(\sigma + 3 i)
X^{\sigma + 3 i} \ d\sigma
$$
\end{definition}


\begin{definition}[I₅]\label{I5}\lean{I₅}\leanok
$$
I_5(\nu, \epsilon, X, \sigma_2) := \frac{1}{2\pi i} \int_{-3}^{3}
\left(
\frac{-\zeta'}\zeta(\sigma_2 + t i)
\right)
  \mathcal M(\widetilde 1_\epsilon)(\sigma_2 + t i)
X^{\sigma_2 + t i} \ i \ dt
$$
\end{definition}


\begin{lemma}[dlog_riemannZeta_bdd_on_vertical_lines]\label{dlog_riemannZeta_bdd_on_vertical_lines}\lean{dlog_riemannZeta_bdd_on_vertical_lines}\leanok
For $\sigma_0 > 1$, there exists a constant $C > 0$ such that
$$
\forall t \in \R, \quad
\left\| \frac{\zeta'(\sigma_0 + t i)}{\zeta(\sigma_0 + t i)} \right\| \leq C.
$$
\end{lemma}


\begin{proof}\uses{LogDerivativeDirichlet}\leanok
Write as Dirichlet series and estimate trivially using Theorem \ref{LogDerivativeDirichlet}.
\end{proof}


\begin{lemma}[SmoothedChebyshevPull1_aux_integrable]\label{SmoothedChebyshevPull1_aux_integrable}\lean{SmoothedChebyshevPull1_aux_integrable}\leanok
The integrand $$\zeta'(s)/\zeta(s)\mathcal{M}(\widetilde{1_{\epsilon}})(s)X^{s}$$
is integrable on the contour $\sigma_0 + t i$ for $t \in \R$ and $\sigma_0 > 1$.
\end{lemma}


\begin{proof}\uses{MellinOfSmooth1b, SmoothedChebyshevDirichlet_aux_integrable}\leanok
The $\zeta'(s)/\zeta(s)$ term is bounded, as is $X^s$, and the smoothing function
$\mathcal{M}(\widetilde{1_{\epsilon}})(s)$
decays like $1/|s|^2$ by Theorem \ref{MellinOfSmooth1b}.
Actually, we already know that
$\mathcal{M}(\widetilde{1_{\epsilon}})(s)$
is integrable from Theorem \ref{SmoothedChebyshevDirichlet_aux_integrable},
so we should just need to bound the rest.
\end{proof}


\begin{lemma}[BddAboveOnRect]\label{BddAboveOnRect}\lean{BddAboveOnRect}\leanok
Let $g : \C \to \C$ be a holomorphic function on a rectangle, then $g$ is bounded above on the rectangle.
\end{lemma}


\begin{proof}\leanok
Use the compactness of the rectangle and the fact that holomorphic functions are continuous.
\end{proof}


\begin{theorem}[SmoothedChebyshevPull1]\label{SmoothedChebyshevPull1}\lean{SmoothedChebyshevPull1}\leanok
We have that
$$\psi_{\epsilon}(X) =
\mathcal{M}(\widetilde{1_{\epsilon}})(1)
X^{1} +
I_1 - I_2 +I_{37} + I_8 + I_9
.
$$
\end{theorem}


\begin{proof}\leanok
\uses{SmoothedChebyshev, RectangleIntegral, ResidueMult, riemannZetaLogDerivResidue,
SmoothedChebyshevPull1_aux_integrable, BddAboveOnRect, BddAbove_to_IsBigO,
I1, I2, I37, I8, I9}
Pull rectangle contours and evaluate the pole at $s=1$.
\end{proof}


Next pull contours to another box.
\begin{lemma}[SmoothedChebyshevPull2]\label{SmoothedChebyshevPull2}\lean{SmoothedChebyshevPull2}\leanok
We have that
$$
I_{37} =
I_3 - I_4 + I_5 + I_6 + I_7
.
$$
\end{lemma}


\begin{proof}\uses{HolomorphicOn.vanishesOnRectangle, I3, I4, I5, I6, I7, I37}\leanok
Mimic the proof of Lemma \ref{SmoothedChebyshevPull1}.
\end{proof}


We insert this information in $\psi_{\epsilon}$. We add and subtract the integral over the box
$[1-\delta,2] \times_{ℂ} [-T,T]$, which we evaluate as follows
\begin{theorem}[ZetaBoxEval]\label{ZetaBoxEval}\lean{ZetaBoxEval}\leanok
For all $\epsilon > 0$ sufficiently close to $0$, the rectangle integral over $[1-\delta,2] \times_{ℂ} [-T,T]$ of the integrand in
$\psi_{\epsilon}$ is
$$
\frac{X^{1}}{1}\mathcal{M}(\widetilde{1_{\epsilon}})(1)
= X(1+O(\epsilon))
,$$
where the implicit constant is independent of $X$.
\end{theorem}


\begin{proof}\leanok
\uses{MellinOfSmooth1c}
Unfold the definitions and apply Lemma \ref{MellinOfSmooth1c}.
\end{proof}


It remains to estimate all of the integrals.


This auxiliary lemma is useful for what follows.
\begin{lemma}[IBound_aux1]\label{IBound_aux1}\lean{IBound_aux1}\leanok
Given a natural number $k$ and a real number $X_0 > 0$, there exists $C \geq 1$ so that for all $X \geq X_0$,
$$
\log^k X \le C \cdot X.
$$
\end{lemma}


\begin{proof}\leanok
We use the fact that $\log^k X / X$ goes to $0$ as $X \to \infty$.
Then we use the extreme value theorem to find a constant $C$ that works for all $X \geq X_0$.
\end{proof}


\begin{lemma}[I1Bound]\label{I1Bound}\lean{I1Bound}\leanok
We have that
$$
\left|I_{1}(\nu, \epsilon, X, T)\
\right| \ll \frac{X}{\epsilon T}
.
$$
Same with $I_9$.
\end{lemma}


\begin{proof}\uses{MellinOfSmooth1b, dlog_riemannZeta_bdd_on_vertical_lines, I1, I9,
  IBound_aux1}\leanok
  Unfold the definitions and apply the triangle inequality.
$$
\left|I_{1}(\nu, \epsilon, X, T)\right| =
\left|
\frac{1}{2\pi i} \int_{-\infty}^{-T}
\left(
\frac{-\zeta'}\zeta(\sigma_0 + t i)
\right)
 \mathcal M(\widetilde 1_\epsilon)(\sigma_0 + t i)
X^{\sigma_0 + t i}
\ i \ dt
\right|
$$
By Theorem \ref{dlog_riemannZeta_bdd_on_vertical_lines} (once fixed!!),
$\zeta'/\zeta (\sigma_0 + t i)$ is bounded by $\zeta'/\zeta(\sigma_0)$, and
Theorem \ref{riemannZetaLogDerivResidue} gives $\ll 1/(\sigma_0-1)$ for the latter. This gives:
$$
\leq
\frac{1}{2\pi}
\left|
 \int_{-\infty}^{-T}
C \log X\cdot
 \frac{C'}{\epsilon|\sigma_0 + t i|^2}
X^{\sigma_0}
\ dt
\right|
,
$$
where we used Theorem \ref{MellinOfSmooth1b}.
Continuing the calculation, we have
$$
\leq
\log X \cdot
C'' \frac{X^{\sigma_0}}{\epsilon}
\int_{-\infty}^{-T}
\frac{1}{t^2}
\ dt
\ \leq \
C''' \frac{X\log X}{\epsilon T}
,
$$
where we used that $\sigma_0=1+1/\log X$, and $X^{\sigma_0} = X\cdot X^{1/\log X}=e \cdot X$.
\end{proof}


\begin{lemma}[I2Bound]\label{I2Bound}\lean{I2Bound}\leanok
Assuming a bound of the form of Lemma \ref{LogDerivZetaBndUnif} we have that
$$
\left|I_{2}(\nu, \epsilon, X, T)\right| \ll \frac{X}{\epsilon T}
.
$$
\end{lemma}


\begin{proof}\uses{MellinOfSmooth1b, I2, I8}\leanok
Unfold the definitions and apply the triangle inequality.
$$
\left|I_{2}(\nu, \epsilon, X, T, \sigma_1)\right| =
\left|\frac{1}{2\pi i} \int_{\sigma_1}^{\sigma_0}
\left(\frac{-\zeta'}\zeta(\sigma - T i) \right) \cdot
\mathcal M(\widetilde 1_\epsilon)(\sigma - T i) \cdot
X^{\sigma - T i}
 \ d\sigma
\right|
$$
$$\leq
\frac{1}{2\pi}
\int_{\sigma_1}^{\sigma_0}
C \cdot \log T ^ 9
\frac{C'}{\epsilon|\sigma - T i|^2}
X^{\sigma_0}
 \ d\sigma
 \leq
C'' \cdot \frac{X\log T^9}{\epsilon T^2}
,
$$
where we used Theorems \ref{MellinOfSmooth1b}, the hypothesised bound on zeta and the fact that
$X^\sigma \le X^{\sigma_0} = X\cdot X^{1/\log X}=e \cdot X$.
Since $T>3$, we have $\log T^9 \leq C''' T$.
\end{proof}


\begin{lemma}[I8I2]\label{I8I2}\lean{I8I2}\leanok
Symmetry between $I_2$ and $I_8$:
$$
I_8(\nu, \epsilon, X, T) = -\overline{I_2(\nu, \epsilon, X, T)}
.
$$
\end{lemma}


\begin{proof}\uses{I2, I8, SmoothedChebyshevIntegrand_conj}\leanok
  This is a direct consequence of the definitions of $I_2$ and $I_8$.
\end{proof}


\begin{lemma}[I8Bound]\label{I8Bound}\lean{I8Bound}\leanok
We have that
$$
\left|I_{8}(\nu, \epsilon, X, T)\right| \ll \frac{X}{\epsilon T}
.
$$
\end{lemma}


\begin{proof}\uses{I8I2, I2Bound}\leanok
  We deduce this from the corresponding bound for $I_2$, using the symmetry between $I_2$ and $I_8$.
\end{proof}


\begin{lemma}[IntegralofLogx^n/x^2Bounded]\label{IntegralofLogx^n/x^2Bounded}\lean{log_pow_over_xsq_integral_bounded}\leanok
For every $n$ there is some absolute constant $C>0$ such that
$$
\int_3^T \frac{(\log x)^9}{x^2}dx < C
$$
\end{lemma}


\begin{proof}\leanok
Induct on n and just integrate by parts.
\end{proof}


\begin{lemma}[I3Bound]\label{I3Bound}\lean{I3Bound}\leanok
Assuming a bound of the form of Lemma \ref{LogDerivZetaBndUnif} we have that
$$
\left|I_{3}(\nu, \epsilon, X, T)\right| \ll \frac{X}{\epsilon}\, X^{-\frac{A}{(\log T)^9}}
.
$$
Same with $I_7$.
\end{lemma}


\begin{proof}\uses{MellinOfSmooth1b, IntegralofLogx^n/x^2Bounded, I3, I7}\leanok
Unfold the definitions and apply the triangle inequality.
$$
\left|I_{3}(\nu, \epsilon, X, T, \sigma_1)\right| =
\left|\frac{1}{2\pi i} \int_{-T}^3
\left(\frac{-\zeta'}\zeta(\sigma_1 + t i) \right)
\mathcal M(\widetilde 1_\epsilon)(\sigma_1 + t i)
X^{\sigma_1 + t i}
\ i \ dt
\right|
$$
$$\leq
\frac{1}{2\pi}
\int_{-T}^3
C \cdot \log t ^ 9
\frac{C'}{\epsilon|\sigma_1 + t i|^2}
X^{\sigma_1}
 \ dt
,
$$
where we used Theorems \ref{MellinOfSmooth1b} and the hypothesised bound on zeta.
Now we estimate $X^{\sigma_1} = X \cdot X^{-A/ \log T^9}$, and the integral is absolutely bounded.
\end{proof}


\begin{lemma}[I4Bound]\label{I4Bound}\lean{I4Bound}\leanok
We have that
$$
\left|I_{4}(\nu, \epsilon, X, \sigma_1, \sigma_2)\right| \ll \frac{X}{\epsilon}\,
 X^{-\frac{A}{(\log T)^9}}
.
$$
Same with $I_6$.
\end{lemma}


\begin{proof}\uses{MellinOfSmooth1b, I4, I6}\leanok
The analysis of $I_4$ is similar to that of $I_2$, (in Lemma \ref{I2Bound}) but even easier.
Let $C$ be the sup of $-\zeta'/\zeta$ on the curve $\sigma_2 + 3 i$ to $1+ 3i$ (this curve is compact, and away from the pole at $s=1$).
Apply Theorem \ref{MellinOfSmooth1b} to get the bound $1/(\epsilon |s|^2)$, which is bounded by $C'/\epsilon$.
And $X^s$ is bounded by $X^{\sigma_1} = X \cdot X^{-A/ \log T^9}$.
Putting these together gives the result.
\end{proof}


\begin{lemma}[I5Bound]\label{I5Bound}\lean{I5Bound}\leanok
We have that
$$
\left|I_{5}(\nu, \epsilon, X, \sigma_2)\right| \ll \frac{X^{\sigma_2}}{\epsilon}.
$$
\end{lemma}


\begin{proof}\uses{MellinOfSmooth1b, LogDerivZetaHolcSmallT, I5}\leanok
Here $\zeta'/\zeta$ is absolutely bounded on the compact interval $\sigma_2 + i [-3,3]$, and
$X^s$ is bounded by $X^{\sigma_2}$. Using Theorem \ref{MellinOfSmooth1b} gives the bound $1/(\epsilon |s|^2)$, which is bounded by $C'/\epsilon$.
Putting these together gives the result.
\end{proof}


\section{MediumPNT}

\begin{theorem}[MediumPNT]\label{MediumPNT}\lean{MediumPNT}\leanok  We have
$$ \sum_{n \leq x} \Lambda(n) = x + O(x \exp(-c(\log x)^{1/10})).$$
\end{theorem}


\begin{proof}
\uses{ChebyshevPsi, SmoothedChebyshevClose, ZetaBoxEval, LogDerivZetaBndUnif, LogDerivZetaHolcSmallT, LogDerivZetaHolcLargeT,
SmoothedChebyshevPull1, SmoothedChebyshevPull2, I1Bound, I2Bound, I3Bound, I4Bound, I5Bound}\leanok
  Evaluate the integrals.
\end{proof}

