
\begin{lemma}[finsum_range_eq_sum_range]\label{finsum_range_eq_sum_range}\lean{finsum_range_eq_sum_range}\leanok For any arithmetic function $f$ and real number $x$, one has
$$ \sum_{n \leq x} f(n) = \sum_{n \leq ⌊x⌋_+} f(n)$$
and
$$ \sum_{n < x} f(n) = \sum_{n < ⌈x⌉_+} f(n).$$
\end{lemma}


\begin{proof}\leanok Straightforward. \end{proof}


\begin{theorem}[chebyshev_asymptotic]\label{chebyshev_asymptotic}\lean{chebyshev_asymptotic}\leanok  One has
  $$ \sum_{p \leq x} \log p = x + o(x).$$
\end{theorem}


\begin{proof}
\uses{WeakPNT, finsum_range_eq_sum_range}\leanok
From the prime number theorem we already have
$$ \sum_{n \leq x} \Lambda(n) = x + o(x)$$
so it suffices to show that
$$ \sum_{j \geq 2} \sum_{p^j \leq x} \log p = o(x).$$
Only the terms with $j \leq \log x / \log 2$ contribute, and each $j$ contributes at most $\sqrt{x} \log x$ to the sum, so the left-hand side is $O( \sqrt{x} \log^2 x ) = o(x)$ as required.
\end{proof}


\begin{corollary}[primorial_bounds]  \label{primorial_bounds}\lean{primorial_bounds}\leanok
We have
  $$ \prod_{p \leq x} p = \exp( x + o(x) )$$
\end{corollary}


\begin{proof}\leanok
\uses{chebyshev_asymptotic}
  Exponentiate Theorem \ref{chebyshev_asymptotic}.
\end{proof}


\begin{theorem}[pi_asymp]\label{pi_asymp}\lean{pi_asymp}\leanok
There exists a function $c(x)$ such that $c(x) = o(1)$ as $x \to \infty$ and
$$ \pi(x) = (1 + c(x)) \int_2^x \frac{dt}{\log t}$$
for all $x$ large enough.
\end{theorem}


\begin{proof}\leanok
\uses{chebyshev_asymptotic}
We have the identity
$$ \pi(x) = \frac{1}{\log x} \sum_{p \leq x} \log p
+ \int_2^x (\sum_{p \leq t} \log p) \frac{dt}{t \log^2 t}$$
as can be proven by interchanging the sum and integral and using the fundamental theorem of calculus.  For any $\eps$, we know from Theorem \ref{chebyshev_asymptotic} that there is $x_\eps$ such that
$\sum_{p \leq t} \log p = t + O(\eps t)$ for $t \geq x_\eps$, hence for $x \geq x_\eps$
$$ \pi(x) = \frac{1}{\log x} (x + O(\eps x))
+ \int_{x_\eps}^x (t + O(\eps t)) \frac{dt}{t \log^2 t} + O_\eps(1)$$
where the $O_\eps(1)$ term can depend on $x_\eps$ but is independent of $x$.  One can evaluate this after an integration by parts as
$$ \pi(x) = (1+O(\eps)) \int_{x_\eps}^x \frac{dt}{\log t} + O_\eps(1)$$
$$  = (1+O(\eps)) \int_{2}^x \frac{dt}{\log t} $$
for $x$ large enough, giving the claim.
\end{proof}


\begin{corollary}[pi_alt]\label{pi_alt}\lean{pi_alt}\leanok  One has
$$ \pi(x) = (1+o(1)) \frac{x}{\log x}$$
as $x \to \infty$.
\end{corollary}


\begin{proof}\leanok
\uses{pi_asymp}
An integration by parts gives
  $$ \int_2^x \frac{dt}{\log t} = \frac{x}{\log x} - \frac{2}{\log 2} + \int_2^x \frac{dt}{\log^2 t}.$$
We have the crude bounds
$$ \int_2^{\sqrt{x}} \frac{dt}{\log^2 t} = O( \sqrt{x} )$$
and
$$ \int_{\sqrt{x}}^x \frac{dt}{\log^2 t} = O( \frac{x}{\log^2 x} )$$
and combining all this we obtain
$$ \int_2^x \frac{dt}{\log t} = \frac{x}{\log x} + O( \frac{x}{\log^2 x} )$$
$$ = (1+o(1)) \frac{x}{\log x}$$
and the claim then follows from Theorem \ref{pi_asymp}.
\end{proof}


Let $p_n$ denote the $n^{th}$ prime.

\begin{proposition}[pn_asymptotic]\label{pn_asymptotic}\lean{pn_asymptotic}\leanok
 One has
  $$ p_n = (1+o(1)) n \log n$$
as $n \to \infty$.
\end{proposition}


\begin{proof}
\uses{pi_alt}\leanok
Use Corollary \ref{pi_alt} to show that for any $\eps>0$, and for $n$ sufficiently large, the number of primes up to $(1-\eps) n \log n$ is less than $n$, and the number of primes up to $(1+\eps) n \log n$ is greater than $n$.
\end{proof}


\begin{corollary}[pn_pn_plus_one] \label{pn_pn_plus_one}\lean{pn_pn_plus_one}\leanok
We have $p_{n+1} - p_n = o(p_n)$
  as $n \to \infty$.
\end{corollary}


\begin{proof}
\uses{pn_asymptotic}\leanok
  Easy consequence of preceding proposition.
\end{proof}


\begin{corollary}[prime_between]  \label{prime_between}\lean{prime_between}\leanok
For every $\eps>0$, there is a prime between $x$ and $(1+\eps)x$ for all sufficiently large $x$.
\end{corollary}


\begin{proof}
\uses{pi_alt}\leanok
Use Corollary \ref{pi_alt} to show that $\pi((1+\eps)x) - \pi(x)$ goes to infinity as $x \to \infty$.
\end{proof}


\begin{proposition}\label{mun}\lean{sum_mobius_div_self_le}\leanok
We have $|\sum_{n \leq x} \frac{\mu(n)}{n}| \leq 1$.
\end{proposition}


\begin{proof}\leanok
From M\"obius inversion $1_{n=1} = \sum_{d|n} \mu(d)$ and summing we have
  $$ 1 = \sum_{d \leq x} \mu(d) \lfloor \frac{x}{d} \rfloor$$
  for any $x \geq 1$. Since $\lfloor \frac{x}{d} \rfloor = \frac{x}{d} - \epsilon_d$ with
  $0 \leq \epsilon_d < 1$ and $\epsilon_x = 0$, we conclude that
  $$ 1 ≥ x \sum_{d \leq x} \frac{\mu(d)}{d} - (x - 1)$$
  and the claim follows.
\end{proof}


\begin{proposition}[M\"obius form of prime number theorem]\label{mu-pnt}\lean{mu_pnt}\leanok  We have $\sum_{n \leq x} \mu(n) = o(x)$.
\end{proposition}


\begin{proof}
\uses{mun, WeakPNT}
From the Dirichlet convolution identity
  $$ \mu(n) \log n = - \sum_{d|n} \mu(d) \Lambda(n/d)$$
and summing we obtain
$$ \sum_{n \leq x} \mu(n) \log n = - \sum_{d \leq x} \mu(d) \sum_{m \leq x/d} \Lambda(m).$$
For any $\eps>0$, we have from the prime number theorem that
$$ \sum_{m \leq x/d} \Lambda(m) = x/d + O(\eps x/d) + O_\eps(1)$$
(divide into cases depending on whether $x/d$ is large or small compared to $\eps$).
We conclude that
$$ \sum_{n \leq x} \mu(n) \log n = - x \sum_{d \leq x} \frac{\mu(d)}{d} + O(\eps x \log x) + O_\eps(x).$$
Applying \eqref{mun} we conclude that
$$ \sum_{n \leq x} \mu(n) \log n = O(\eps x \log x) + O_\eps(x).$$
and hence
$$ \sum_{n \leq x} \mu(n) \log x = O(\eps x \log x) + O_\eps(x) + O( \sum_{n \leq x} (\log x - \log n) ).$$
From Stirling's formula one has
$$  \sum_{n \leq x} (\log x - \log n) = O(x)$$
thus
$$ \sum_{n \leq x} \mu(n) \log x = O(\eps x \log x) + O_\eps(x)$$
and thus
$$ \sum_{n \leq x} \mu(n) = O(\eps x) + O_\eps(\frac{x}{\log x}).$$
Sending $\eps \to 0$ we obtain the claim.
\end{proof}


\begin{proposition}\label{lambda-pnt}\lean{lambda_pnt}\leanok
We have $\sum_{n \leq x} \lambda(n) = o(x)$.
\end{proposition}


\begin{proof}
\uses{mu-pnt}
From the identity
  $$ \lambda(n) = \sum_{d^2|n} \mu(n/d^2)$$
and summing, we have
$$ \sum_{n \leq x} \lambda(n) = \sum_{d \leq \sqrt{x}} \sum_{n \leq x/d^2} \mu(n).$$
For any $\eps>0$, we have from Proposition \ref{mu-pnt} that
$$ \sum_{n \leq x/d^2} \mu(n) = O(\eps x/d^2) + O_\eps(1)$$
and hence on summing in $d$
$$ \sum_{n \leq x} \lambda(n) = O(\eps x) + O_\eps(x^{1/2}).$$
Sending $\eps \to 0$ we obtain the claim.
\end{proof}



\begin{proposition}[Alternate M\"obius form of prime number theorem]\label{mu-pnt-alt}\lean{mu_pnt_alt}\leanok  We have $\sum_{n \leq x} \mu(n)/n = o(1)$.
\end{proposition}


\begin{proof}
\uses{mu-pnt}
As in the proof of Theorem \ref{mun}, we have
  $$ 1 = \sum_{d \leq x} \mu(d) \lfloor \frac{x}{d} \rfloor$$
  $$ = x \sum_{d \leq x} \frac{\mu(d)}{d} - \sum_{d \leq x} \mu(d) \{ \frac{x}{d} \}$$
so it will suffice to show that
$$ \sum_{d \leq x} \mu(d) \{ \frac{x}{d} \} = o(x).$$
Let $N$  be a natural number.  It suffices to show that
$$ \sum_{d \leq x} \mu(d) \{ \frac{x}{d} \} = O(x/N).$$
if $x$ is large enough depending on $N$.
We can split the left-hand side as the sum of
$$ \sum_{d \leq x/N} \mu(d) \{ \frac{x}{d} \} $$
and
$$ \sum_{j=1}^{N-1} \sum_{x/(j+1) < d \leq x/j} \mu(d) (x/d - j).$$
The first term is clearly $O(x/N)$.  For the second term, we can use Theorem \ref{mu-pnt} and summation by parts (using the fact that $x/d-j$ is monotone and bounded) to find that
$$ \sum_{x/(j+1) < d \leq x/j} \mu(d) (x/d - j) = o(x)$$
for any given $j$, so in particular
$$ \sum_{x/(j+1) < d \leq x/j} \mu(d) (x/d - j) = O(x/N^2)$$
for all $j=1,\dots,N-1$ if $x$ is large enough depending on $N$.  Summing all the bounds, we obtain the claim.
\end{proof}


\section{Consequences of the PNT in arithmetic progressions}

\begin{theorem}[Prime number theorem in AP]\label{chebyshev_asymptotic_pnt}\lean{chebyshev_asymptotic_pnt}\leanok  If $a\ (q)$ is a primitive residue class, then one has
  $$ \sum_{p \leq x: p = a\ (q)} \log p = \frac{x}{\phi(q)} + o(x).$$
\end{theorem}


\begin{proof}
\uses{chebyshev_asymptotic}
This is a routine modification of the proof of Theorem \ref{chebyshev_asymptotic}.
\end{proof}


\begin{corollary}[Dirichlet's theorem]\label{dirichlet_thm}\lean{dirichlet_thm}\leanok  Any primitive residue class contains an infinite number of primes.
\end{corollary}


\begin{proof}
\uses{chebyshev_asymptotic_pnt}
If this were not the case, then the sum $\sum_{p \leq x: p = a\ (q)} \log p$ would be bounded in $x$, contradicting Theorem \ref{chebyshev_asymptotic_pnt}.
\end{proof}
-/

/-%%
\section{Consequences of the Chebotarev density theorem}



\begin{lemma}[Cyclotomic Chebotarev]\label{Chebotarev-cyclic}  For any $a$ coprime to $m$,
$$ \sum_{N \mathfrak{p} \leq x; N \mathfrak{p} = a\ (m)} \log N \mathfrak{p}  =
\frac{1}{|G|} \sum_{N \mathfrak{p} \leq x} \log N \mathfrak{p}.$$
\end{lemma}


\begin{proof}\uses{Dedekind-PNT, WeakPNT_AP} This should follow from Lemma \ref{Dedekind-PNT} by a Fourier expansion.
\end{proof}

