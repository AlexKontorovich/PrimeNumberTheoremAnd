
\begin{theorem}[hasDerivAt_conj_conj]\label{hasDerivAt_conj_conj}\lean{hasDerivAt_conj_conj}\leanok
Let $f : \mathbb{C} \to \mathbb{C}$ be a complex differentiable function at $p \in \mathbb{C}$ with derivative $a$.
Then the function $g(z) = \overline{f(\overline{z})}$ is complex differentiable at $\overline{p}$ with derivative $\overline{a}$.
\end{theorem}
Already on Mathlib (with a shortened proof).


\begin{proof}\leanok
We expand the definition of the derivative and compute.
\end{proof}


\begin{theorem}[deriv_conj_conj]\label{deriv_conj_conj}\lean{deriv_conj_conj}\leanok
Let $f : \mathbb{C} \to \mathbb{C}$ be a function at $p \in \mathbb{C}$ with derivative $a$.
Then the derivative of the function $g(z) = \overline{f(\overline{z})}$ at $\overline{p}$ is $\overline{a}$.
\end{theorem}
Submitted to Mathlib.


\begin{proof}\uses{hasDerivAt_conj_conj}\leanok
We proceed by case analysis on whether $f$ is differentiable at $p$.
If $f$ is differentiable at $p$, then we can apply the previous theorem.
If $f$ is not differentiable at $p$, then neither is $g$, and both derivatives have the default value of zero.
\end{proof}


\begin{theorem}[conj_riemannZeta_conj_aux1]\label{conj_riemannZeta_conj_aux1}\lean{conj_riemannZeta_conj_aux1}\leanok
Conjugation symmetry of the Riemann zeta function in the half-plane of convergence.
Let $s \in \mathbb{C}$ with $\Re(s) > 1$.
Then $\overline{\zeta(\overline{s})} = \zeta(s)$.
\end{theorem}


\begin{proof}\leanok
We expand the definition of the Riemann zeta function as a series and find that the two sides are equal term by term.
\end{proof}


\begin{theorem}[conj_riemannZeta_conj]\label{conj_riemannZeta_conj}\lean{conj_riemannZeta_conj}\leanok
Conjugation symmetry of the Riemann zeta function.
Let $s \in \mathbb{C}$.
Then $$\overline{\zeta(\overline{s})} = \zeta(s).$$
\end{theorem}

% TODO: Submit this and the following corollaries to Mathlib.


\begin{proof}\uses{conj_riemannZeta_conj_aux1}\leanok
By the previous lemma, the two sides are equal on the half-plane $\{s \in \mathbb{C} : \Re(s) > 1\}$. Then, by analytic continuation, they are equal on the whole complex plane.
\end{proof}


\begin{theorem}[riemannZeta_conj]\label{riemannZeta_conj}\lean{riemannZeta_conj}\leanok
Conjugation symmetry of the Riemann zeta function.
Let $s \in \mathbb{C}$.
Then $$\zeta(\overline{s}) = \overline{\zeta(s)}.$$
\end{theorem}


\begin{proof}\leanok
This follows as an immediate corollary of Theorem \ref{conj_riemannZeta_conj}.
\end{proof}


\begin{theorem}[deriv_riemannZeta_conj]\label{deriv_riemannZeta_conj}\lean{deriv_riemannZeta_conj}\leanok
Conjugation symmetry of the derivative of the Riemann zeta function.
Let $s \in \mathbb{C}$.
Then $$\zeta'(\overline{s}) = \overline{\zeta'(s)}.$$
\end{theorem}


\begin{proof}\leanok
We apply the derivative conjugation symmetry to the Riemann zeta function and use the conjugation symmetry of the Riemann zeta function itself.
\end{proof}


\begin{theorem}[intervalIntegral_conj]\label{intervalIntegral_conj}\lean{intervalIntegral_conj}\leanok
The conjugation symmetry of the interval integral.
Let $f : \mathbb{R} \to \mathbb{C}$ be a measurable function, and let $a, b \in \mathbb{R}$.
Then $$\int_{a}^{b} \overline{f(x)} \, dx = \overline{\int_{a}^{b} f(x) \, dx}.$$
\end{theorem}

% TODO: Submit this to Mathlib.


\begin{proof}\leanok
We unfold the interval integral into an integral over a uIoc and use the conjugation property of integrals.
\end{proof}

