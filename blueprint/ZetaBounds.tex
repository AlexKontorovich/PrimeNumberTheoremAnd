
We record here some prelimiaries about the zeta function and general
holomorphic functions.

\begin{theorem}[ResidueOfTendsTo]\label{ResidueOfTendsTo}\lean{ResidueOfTendsTo}\leanok
  If a function $f$ is holomorphic in a neighborhood of $p$ and
  $\lim_{s\to p} (s-p)f(s) = A$, then
  $f(s) = \frac{A}{s-p} + O(1)$ near $p$.
\end{theorem}


\begin{proof}\uses{existsDifferentiableOn_of_bddAbove}\leanok
The function $(s - p)\cdot f(s)$ bounded, so by Theorem
\ref{existsDifferentiableOn_of_bddAbove}, there is a holomorphic function, $g$, say, so that
$(s-p)f(s) = g(s)$ in a neighborhood of $s=p$, and $g(p)=A$. Now because $g$ is holomorphic,
near $s=p$, we have $g(s)=A+O(s-p)$. Then when you divide by $(s-p)$, you get
$f(s) = A/(s-p) + O(1)$.
\end{proof}


\begin{theorem}[riemannZetaResidue]\label{riemannZetaResidue}\lean{riemannZetaResidue}\leanok
  The Riemann zeta function $\zeta(s)$ has a simple pole at $s=1$ with residue $1$. In particular, the function
  $$ \zeta(s) - \frac{1}{s-1}$$
  is bounded in a neighborhood of $s=1$.
\end{theorem}


\begin{proof}\uses{ResidueOfTendsTo}\leanok
From `riemannZeta_residue_one` (in Mathlib), we know that
$(s-1)\zeta(s)$ goes to $1$ as $s\to1$. Now apply Theorem \ref{ResidueOfTendsTo}.
(This can also be done using $\zeta_0$ below, which is expressed as
$1/(s-1)$ plus things that are holomorphic for $\Re(s)>0$...)
\end{proof}


\begin{theorem}[nonZeroOfBddAbove]\label{nonZeroOfBddAbove}\lean{nonZeroOfBddAbove}\leanok
  If a function $f$ has a simple pole at a point $p$ with residue $A \neq 0$, then
  $f$ is nonzero in a punctured neighborhood of $p$.
\end{theorem}


  \begin{proof}\leanok
    We know that $f(s) = \frac{A}{s-p} + O(1)$ near $p$, so we can write
    $$f(s) = \left(f(s) - \frac{A}{s-p}\right) + \frac{A}{s-p}.$$
    The first term is bounded, say by $M$, and the second term goes to $\infty$ as $s \to p$.
    Therefore, there exists a neighborhood $V$ of $p$ such that for all $s \in V \setminus \{p\}$,
    we have $f(s) \neq 0$.
  \end{proof}
  

\begin{theorem}[logDerivResidue]\label{logDerivResidue}\lean{logDerivResidue}\leanok
  If $f$ is holomorphic in a neighborhood of $p$, and there is a simple pole at $p$, then $f'/
  f$ has a simple pole at $p$ with residue $-1$:
  $$ \frac{f'(s)}{f(s)} = \frac{-1}{s - p} + O(1).$$
\end{theorem}


\begin{proof}\uses{existsDifferentiableOn_of_bddAbove}\leanok
Using Theorem \ref{existsDifferentiableOn_of_bddAbove}, there is a function $g$ holomorphic  near $p$, for which $f(s) = A/(s-p) + g(s) = h(s)/ (s-p)$. Here $h(s):= A + g(s)(s-p)$ which is nonzero in a neighborhood of $p$ (since $h$ goes to $A$ which is nonzero).
Then $f'(s) = (h'(s)(s-p) - h(s))/(s-p)^2$, and we can compute the quotient:
$$
\frac{f'(s)}{f(s)}+1/(s-p) = \frac{h'(s)(s-p) - h(s)}{h(s)} \cdot \frac{1}{(s-p)}+1/(s-p)
=
\frac{h'(s)}{h(s)}.
$$
Since $h$ is nonvanishing near $p$, this remains bounded in a neighborhood of $p$.
\end{proof}


\begin{theorem}[BddAbove_to_IsBigO]\label{BddAbove_to_IsBigO}\lean{BddAbove_to_IsBigO}\leanok
  If $f$ is bounded above in a punctured neighborhood of $p$, then $f$ is $O(1)$ in that neighborhood.
\end{theorem}


\begin{proof}\leanok
Elementary.
\end{proof}


Let's also record that if a function $f$ has a simple pole at $p$ with residue $A$, and $g$ is holomorphic near $p$, then the residue of $f \cdot g$ is $A \cdot g(p)$.
\begin{theorem}[ResidueMult]\label{ResidueMult}\lean{ResidueMult}\leanok
  If $f$ has a simple pole at $p$ with residue $A$, and $g$ is holomorphic near $p$, then the residue of $f \cdot g$ at $p$ is $A \cdot g(p)$. That is, we assume that
  $$
  f(s) = \frac{A}{s - p} + O(1)$$
  near $p$, and that $g$ is holomorphic near $p$. Then
  $$
  f(s) \cdot g(s) = \frac{A \cdot g(p)}{s - p} + O(1).$$
\end{theorem}


\begin{proof}\leanok
Elementary calculation.
$$
f(s) * g(s) - \frac{A * g(p)}{s - p} =
\left(f(s) * g(s) - \frac{A * g(s)}{s - p}\right) + \left(\frac{A * g(s) - A * g(p)}{s - p}\right).
$$
The first term is $g(s)(f(s) - \frac{A}{s - p})$, which is bounded near $p$ by the assumption on $f$
 and the fact that $g$ is holomorphic near $p$.
The second term is $A$ times the log derivative of $g$ at $p$, which is bounded by the assumption
that  $g$ is holomorphic.
\end{proof}


As a corollary, the log derivative of the Riemann zeta function has a simple pole at $s=1$:
\begin{theorem}[riemannZetaLogDerivResidue]\label{riemannZetaLogDerivResidue}\lean{riemannZetaLogDerivResidue}\leanok
  The log derivative of the Riemann zeta function $\zeta(s)$ has a simple pole at $s=1$ with residue $-1$:
  $$ -\frac{\zeta'(s)}{\zeta(s)} - \frac{1}{s-1} = O(1).$$
\end{theorem}


\begin{proof}\uses{logDerivResidue, riemannZetaResidue, nonZeroOfBddAbove}\leanok
  This follows from Theorem \ref{logDerivResidue} and Theorem \ref{riemannZetaResidue}.
\end{proof}


\begin{definition}[riemannZeta0]\label{riemannZeta0}\lean{riemannZeta0}\leanok
For any natural $N\ge1$, we define
$$
\zeta_0(N,s) :=
\sum_{1\le n \le N} \frac1{n^s}
+
\frac{- N^{1-s}}{1-s} + \frac{-N^{-s}}{2} + s \int_N^\infty \frac{\lfloor x\rfloor + 1/2 - x}{x^{s+1}} \, dx
$$
\end{definition}


\begin{lemma}[sum_eq_int_deriv]\label{sum_eq_int_deriv}\lean{sum_eq_int_deriv}\leanok
  Let $a < b$, and let $\phi$ be continuously differentiable on $[a, b]$.
  Then
  \[
  \sum_{a < n \le b} \phi(n) = \int_a^b \phi(x) \, dx + \left(\lfloor b \rfloor + \frac{1}{2} - b\right) \phi(b) - \left(\lfloor a \rfloor + \frac{1}{2} - a\right) \phi(a) - \int_a^b \left(\lfloor x \rfloor + \frac{1}{2} - x\right) \phi'(x) \, dx.
  \]
\end{lemma}


\begin{proof}\leanok
Specialize Abel summation from Mathlib to the trivial arithmetic function and then manipulate integrals.
\end{proof}


\begin{lemma}[ZetaSum_aux1]\label{ZetaSum_aux1}\lean{ZetaSum_aux1}\leanok
  Let $0 < a < b$ be natural numbers and $s\in \C$ with $s \ne 1$ and $s \ne 0$.
  Then
  \[
  \sum_{a < n \le b} \frac{1}{n^s} =  \frac{b^{1-s} - a^{1-s}}{1-s} + \frac{b^{-s}-a^{-s}}{2} + s \int_a^b \frac{\lfloor x\rfloor + 1/2 - x}{x^{s+1}} \, dx.
  \]
\end{lemma}


\begin{proof}\uses{sum_eq_int_deriv}\leanok
  Apply Lemma \ref{sum_eq_int_deriv} to the function $x \mapsto x^{-s}$.
\end{proof}


\begin{lemma}[ZetaBnd_aux1a]\label{ZetaBnd_aux1a}\lean{ZetaBnd_aux1a}\leanok
For any $0 < a < b$ and  $s \in \C$ with $\sigma=\Re(s)>0$,
$$
\int_a^b \left|\frac{\lfloor x\rfloor + 1/2 - x}{x^{s+1}} \, dx\right|
\le \frac{a^{-\sigma}-b^{-\sigma}}{\sigma}.
$$
\end{lemma}


\begin{proof}\leanok
Apply the triangle inequality
$$
\left|\int_a^b \frac{\lfloor x\rfloor + 1/2 - x}{x^{s+1}} \, dx\right|
\le \int_a^b \frac{1}{x^{\sigma+1}} \, dx,
$$
and evaluate the integral.
\end{proof}


\begin{lemma}[ZetaSum_aux2]\label{ZetaSum_aux2}\lean{ZetaSum_aux2}\leanok
  Let $N$ be a natural number and $s\in \C$, $\Re(s)>1$.
  Then
  \[
  \sum_{N < n} \frac{1}{n^s} =  \frac{- N^{1-s}}{1-s} + \frac{-N^{-s}}{2} + s \int_N^\infty \frac{\lfloor x\rfloor + 1/2 - x}{x^{s+1}} \, dx.
  \]
\end{lemma}


\begin{proof}\uses{ZetaSum_aux1}\leanok
  Apply Lemma \ref{ZetaSum_aux1} with $a=N$ and $b\to \infty$.
\end{proof}


\begin{lemma}[ZetaBnd_aux1b]\label{ZetaBnd_aux1b}\lean{ZetaBnd_aux1b}\leanok
For any $N\ge1$ and $s = \sigma + tI \in \C$, $\sigma > 0$,
$$
\left| \int_N^\infty \frac{\lfloor x\rfloor + 1/2 - x}{x^{s+1}} \, dx \right|
\le \frac{N^{-\sigma}}{\sigma}.
$$
\end{lemma}


\begin{proof}\uses{ZetaBnd_aux1a}\leanok
Apply Lemma \ref{ZetaBnd_aux1a} with $a=N$ and $b\to \infty$.
\end{proof}


\begin{lemma}[ZetaBnd_aux1]\label{ZetaBnd_aux1}\lean{ZetaBnd_aux1}\leanok
For any $N\ge1$ and $s = \sigma + tI \in \C$, $\sigma=\in(0,2], 2 < |t|$,
$$
\left| s\int_N^\infty \frac{\lfloor x\rfloor + 1/2 - x}{x^{s+1}} \, dx \right|
\le 2 |t| \frac{N^{-\sigma}}{\sigma}.
$$
\end{lemma}


\begin{proof}\uses{ZetaBnd_aux1b}\leanok
Apply Lemma \ref{ZetaBnd_aux1b} and estimate $|s|\ll |t|$.
\end{proof}


Big-Oh version of Lemma \ref{ZetaBnd_aux1}.
\begin{lemma}[ZetaBnd_aux1p]\label{ZetaBnd_aux1p}\lean{ZetaBnd_aux1p}\leanok
For any $N\ge1$ and $s = \sigma + tI \in \C$, $\sigma=\in(0,2], 2 < |t|$,
$$
\left| s\int_N^\infty \frac{\lfloor x\rfloor + 1/2 - x}{x^{s+1}} \, dx \right|
\ll |t| \frac{N^{-\sigma}}{\sigma}.
$$
\end{lemma}


\begin{proof}\uses{ZetaBnd_aux1b}\leanok
Apply Lemma \ref{ZetaBnd_aux1b} and estimate $|s|\ll |t|$.
\end{proof}


\begin{lemma}[HolomorphicOn_Zeta0]\label{HolomorphicOn_Zeta0}\lean{HolomorphicOn_Zeta0}\leanok
For any $N\ge1$, the function $\zeta_0(N,s)$ is holomorphic on $\{s\in \C\mid \Re(s)>0 ∧ s \ne 1\}$.
\end{lemma}


\begin{proof}\uses{riemannZeta0, ZetaBnd_aux1b}\leanok
  The function $\zeta_0(N,s)$ is a finite sum of entire functions, plus an integral
  that's absolutely convergent on $\{s\in \C\mid \Re(s)>0 ∧ s \ne 1\}$ by Lemma \ref{ZetaBnd_aux1b}.
\end{proof}


\begin{lemma}[isPathConnected_aux]\label{isPathConnected_aux}\lean{isPathConnected_aux}\leanok
The set $\{s\in \C\mid \Re(s)>0 ∧ s \ne 1\}$ is path-connected.
\end{lemma}


\begin{proof}\leanok
  Construct explicit paths from $2$ to any point, either a line segment or two joined ones.
\end{proof}


\begin{lemma}[Zeta0EqZeta]\label{Zeta0EqZeta}\lean{Zeta0EqZeta}\leanok
For $\Re(s)>0$, $s\ne1$, and for any $N$,
$$
\zeta_0(N,s) = \zeta(s).
$$
\end{lemma}


\begin{proof}\leanok
\uses{ZetaSum_aux2, riemannZeta0, HolomorphicOn_Zeta0, isPathConnected_aux}
Use Lemma \ref{ZetaSum_aux2} and the Definition \ref{riemannZeta0}.
\end{proof}


\begin{lemma}[ZetaBnd_aux2]\label{ZetaBnd_aux2}\lean{ZetaBnd_aux2}\leanok
Given $n ≤ t$ and $\sigma$ with $1-A/\log t \le \sigma$, we have
that
$$
|n^{-s}| \le n^{-1} e^A.
$$
\end{lemma}


\begin{proof}\leanok
Use $|n^{-s}| = n^{-\sigma}
= e^{-\sigma \log n}
\le
\exp(-\left(1-\frac{A}{\log t}\right)\log n)
\le
n^{-1} e^A$,
since $n\le t$.
\end{proof}


\begin{lemma}[ZetaUpperBnd]\label{ZetaUpperBnd}\lean{ZetaUpperBnd}\leanok
For any $s = \sigma + tI \in \C$, $1/2 \le \sigma\le 2, 3 < |t|$
and any $0 < A < 1$ sufficiently small, and $1-A/\log |t| \le \sigma$, we have
$$
|\zeta(s)| \ll \log t.
$$
\end{lemma}


\begin{proof}\uses{ZetaBnd_aux1, ZetaBnd_aux2, Zeta0EqZeta}\leanok
First replace $\zeta(s)$ by $\zeta_0(N,s)$ for $N = \lfloor |t| \rfloor$.
We estimate:
$$
|\zeta_0(N,s)| \ll
\sum_{1\le n \le |t|} |n^{-s}|
+
\frac{- |t|^{1-\sigma}}{|1-s|} + \frac{-|t|^{-\sigma}}{2} +
|t| \cdot |t| ^ {-σ} / σ
$$
$$
\ll
e^A \sum_{1\le n < |t|} n^{-1}
+|t|^{1-\sigma}
$$
,
where we used Lemma \ref{ZetaBnd_aux2} and Lemma \ref{ZetaBnd_aux1}.
The first term is $\ll \log |t|$.
For the second term, estimate
$$
|t|^{1-\sigma}
\le |t|^{1-(1-A/\log |t|)}
= |t|^{A/\log |t|} \ll 1.
$$
\end{proof}


\begin{lemma}[DerivUpperBnd_aux7]\label{DerivUpperBnd_aux7}\lean{DerivUpperBnd_aux7}\leanok
For any $s = \sigma + tI \in \C$, $1/2 \le \sigma\le 2, 3 < |t|$, and any $0 < A < 1$ sufficiently small,
and $1-A/\log |t| \le \sigma$, we have
$$
\left\|s \cdot \int_{N}^{\infty} \left(\left\lfloor x \right\rfloor + \frac{1}{2} - x\right) \cdot x^{-s-1} \cdot (-\log x)\right\|
\le 2 \cdot |t| \cdot N^{-\sigma} / \sigma \cdot \log |t|.
$$
\end{lemma}


\begin{proof}\leanok
Estimate $|s|= |\sigma + tI|$ by $|s|\le 2 +|t| \le 2|t|$ (since $|t|>3$). Estimating $|\left\lfloor x \right\rfloor+1/2-x|$ by $1$,
and using $|x^{-s-1}| = x^{-\sigma-1}$, we have
$$
\left\| s \cdot \int_{N}^{\infty} \left(\left\lfloor x \right\rfloor + \frac{1}{2} - x\right) \cdot x^{-s-1} \cdot (-\log x)\right\|
\le 2 \cdot |t|
\int_{N}^{\infty} x^{-\sigma} \cdot (\log x).
$$
For the last integral, integrate by parts, getting:
$$
\int_{N}^{\infty} x^{-\sigma-1} \cdot (\log x) =
\frac{1}{\sigma}N^{-\sigma} \cdot \log N + \frac1{\sigma^2} \cdot N^{-\sigma}.
$$
Now use $\log N \le \log |t|$ to get the result.
\end{proof}


\begin{lemma}[ZetaDerivUpperBnd]\label{ZetaDerivUpperBnd}\lean{ZetaDerivUpperBnd}\leanok
For any $s = \sigma + tI \in \C$, $1/2 \le \sigma\le 2, 3 < |t|$,
there is an $A>0$ so that for $1-A/\log t \le \sigma$, we have
$$
|\zeta'(s)| \ll \log^2 t.
$$
\end{lemma}


\begin{proof}\uses{ZetaBnd_aux1, ZetaBnd_aux2, Zeta0EqZeta, DerivUpperBnd_aux7}\leanok
First replace $\zeta(s)$ by $\zeta_0(N,s)$ for $N = \lfloor |t| \rfloor$.
Differentiating term by term, we get:
$$
\zeta'(s) = -\sum_{1\le n < N} n^{-s} \log n
+ \frac{N^{1 - s}}{(1 - s)^2} + \frac{N^{1 - s} \log N} {1 - s}
+ \frac{N^{-s}\log N}{2} +
\int_N^\infty \frac{\lfloor x\rfloor + 1/2 - x}{x^{s+1}} \, dx
-s \int_N^\infty \log x \frac{\lfloor x\rfloor + 1/2 - x}{x^{s+1}} \, dx
.
$$
Estimate as before, with an extra factor of $\log |t|$.
\end{proof}


\begin{lemma}[ZetaNear1BndFilter]\label{ZetaNear1BndFilter}\lean{ZetaNear1BndFilter}\leanok
As $\sigma\to1^+$,
$$
|\zeta(\sigma)| \ll 1/(\sigma-1).
$$
\end{lemma}


\begin{proof}\uses{ZetaBnd_aux1, Zeta0EqZeta}\leanok
Zeta has a simple pole at $s=1$. Equivalently, $\zeta(s)(s-1)$ remains bounded near $1$.
Lots of ways to prove this.
Probably the easiest one: use the expression for $\zeta_0 (N,s)$ with $N=1$ (the term $N^{1-s}/(1-s)$ being the only unbounded one).
\end{proof}


\begin{lemma}[ZetaNear1BndExact]\label{ZetaNear1BndExact}\lean{ZetaNear1BndExact}\leanok
There exists a $c>0$ such that for all $1 < \sigma ≤ 2$,
$$
|\zeta(\sigma)| ≤ c/(\sigma-1).
$$
\end{lemma}


\begin{proof}\uses{ZetaNear1BndFilter}\leanok
Split into two cases, use Lemma \ref{ZetaNear1BndFilter} for $\sigma$ sufficiently small
and continuity on a compact interval otherwise.
\end{proof}


\begin{lemma}[ZetaLowerBound3]\label{ZetaLowerBound3}\lean{ZetaLowerBound3}\leanok
There exists a $c>0$ such that for all $1 < \sigma <= 2$ and $3 < |t|$,
$$
c \frac{(\sigma-1)^{3/4}}{(\log |t|)^{1/4}} \le |\zeta(\sigma + tI)|.
$$
\end{lemma}


\begin{proof}\uses{ZetaUpperBnd, ZetaNear1BndExact}\leanok
Combine Lemma \ref{ZetaLowerBound2} with upper bounds for
$|\zeta(\sigma)|$ (from Lemma \ref{ZetaNear1BndExact}) and
$|\zeta(\sigma+2it)|$ (from Lemma \ref{ZetaUpperBnd}).
\end{proof}


\begin{lemma}[ZetaInvBound1]\label{ZetaInvBound1}\lean{ZetaInvBound1}\leanok
For all $\sigma>1$,
$$
1/|\zeta(\sigma+it)| \le |\zeta(\sigma)|^{3/4}|\zeta(\sigma+2it)|^{1/4}
$$
\end{lemma}


\begin{proof}\leanok
The identity
$$
1 \le |\zeta(\sigma)|^3 |\zeta(\sigma+it)|^4 |\zeta(\sigma+2it)|
$$
for $\sigma>1$
is already proved by Michael Stoll in the EulerProducts PNT file.
\end{proof}


\begin{lemma}[ZetaInvBound2]\label{ZetaInvBound2}\lean{ZetaInvBound2}\leanok
For $\sigma>1$ (and $\sigma \le 2$),
$$
1/|\zeta(\sigma+it)| \ll (\sigma-1)^{-3/4}(\log |t|)^{1/4},
$$
as $|t|\to\infty$.
\end{lemma}


\begin{proof}\uses{ZetaInvBound1, ZetaNear1BndExact, ZetaUpperBnd}\leanok
Combine Lemma \ref{ZetaInvBound1} with the bounds in Lemmata \ref{ZetaNear1BndExact} and
\ref{ZetaUpperBnd}.
\end{proof}


\begin{lemma}[Zeta_eq_int_derivZeta]\label{Zeta_eq_int_derivZeta}\lean{Zeta_eq_int_derivZeta}
\leanok
For any $t\ne0$ (so we don't pass through the pole), and $\sigma_1 < \sigma_2$,
$$
\int_{\sigma_1}^{\sigma_2}\zeta'(\sigma + it) dt =
\zeta(\sigma_2+it) - \zeta(\sigma_1+it).
$$
\end{lemma}


\begin{proof}\leanok
This is the fundamental theorem of calculus.
\end{proof}


\begin{lemma}[Zeta_diff_Bnd]\label{Zeta_diff_Bnd}\lean{Zeta_diff_Bnd}\leanok
For any $A>0$ sufficiently small, there is a constant $C>0$ so that
whenever $1- A / \log t \le \sigma_1 < \sigma_2\le 2$ and $3 < |t|$, we have that:
$$
|\zeta (\sigma_2 + it) - \zeta (\sigma_1 + it)|
\le C (\log |t|)^2 (\sigma_2 - \sigma_1).
$$
\end{lemma}


\begin{proof}
\uses{Zeta_eq_int_derivZeta, ZetaDerivUpperBnd}\leanok
Use Lemma \ref{Zeta_eq_int_derivZeta} and
estimate trivially using Lemma \ref{ZetaDerivUpperBnd}.
\end{proof}


\begin{lemma}[ZetaInvBnd]\label{ZetaInvBnd}\lean{ZetaInvBnd}\leanok
For any $A>0$ sufficiently small, there is a constant $C>0$ so that
whenever $1- A / \log^9 |t| \le \sigma < 1+A/\log^9 |t|$ and $3 < |t|$, we have that:
$$
1/|\zeta(\sigma+it)| \le C \log^7 |t|.
$$
\end{lemma}


\begin{proof}\leanok
\uses{Zeta_diff_Bnd, ZetaInvBound2}
Let $\sigma$ be given in the prescribed range, and set $\sigma' := 1+ A / \log^9 |t|$.
Then
$$
|\zeta(\sigma+it)| \ge
|\zeta(\sigma'+it)| - |\zeta(\sigma+it) - \zeta(\sigma'+it)|
\ge
C (\sigma'-1)^{3/4}\log |t|^{-1/4} - C \log^2 |t| (\sigma'-\sigma)
$$
$$
\ge
C A^{3/4} \log |t|^{-7} - C \log^2 |t| (2 A / \log^9 |t|),
$$
where we used Lemma \ref{ZetaInvBound2}  and Lemma \ref{Zeta_diff_Bnd}.
Now by making $A$ sufficiently small (in particular, something like $A = 1/16$ should work), we can guarantee that
$$
|\zeta(\sigma+it)| \ge \frac C 2 (\log |t|)^{-7},
$$
as desired.
\end{proof}


Annoyingly, it is not immediate from this that $\zeta$ doesn't vanish there! That's because
$1/0 = 0$ in Lean. So we give a second proof of the same fact (refactor this later), with a lower
 bound on $\zeta$ instead of upper bound on $1 / \zeta$.
\begin{lemma}[ZetaLowerBnd]\label{ZetaLowerBnd}\lean{ZetaLowerBnd}\leanok
For any $A>0$ sufficiently small, there is a constant $C>0$ so that
whenever $1- A / \log^9 |t| \le \sigma < 1$ and $3 < |t|$, we have that:
$$
|\zeta(\sigma+it)| \ge C \log^7 |t|.
$$
\end{lemma}


\begin{proof}\leanok
\uses{ZetaLowerBound3, Zeta_diff_Bnd}
Follow same argument.
\end{proof}


Now we get a zero free region.
\begin{lemma}[ZetaZeroFree]\label{ZetaZeroFree}\lean{ZetaZeroFree}\leanok
There is an $A>0$ so that for $1-A/\log^9 |t| \le \sigma < 1$ and $3 < |t|$,
$$
\zeta(\sigma+it) \ne 0.
$$
\end{lemma}


\begin{proof}\leanok
\uses{ZetaLowerBnd}
Apply Lemma \ref{ZetaLowerBnd}.
\end{proof}


\begin{lemma}[LogDerivZetaBnd]\label{LogDerivZetaBnd}\lean{LogDerivZetaBnd}\leanok
There is an $A>0$ so that for $1-A/\log^9 |t| \le \sigma < 1+A/\log^9 |t|$ and $3 < |t|$,
$$
|\frac {\zeta'}{\zeta} (\sigma+it)| \ll \log^9 |t|.
$$
\end{lemma}


\begin{proof}\leanok
\uses{ZetaInvBnd, ZetaDerivUpperBnd}
Combine the bound on $|\zeta'|$ from Lemma \ref{ZetaDerivUpperBnd} with the
bound on $1/|\zeta|$ from Lemma \ref{ZetaInvBnd}.
\end{proof}


\begin{theorem}[ZetaNoZerosOn1Line]\label{ZetaNoZerosOn1Line}\lean{ZetaNoZerosOn1Line}\leanok
The zeta function does not vanish on the 1-line.
\end{theorem}


\begin{proof}\leanok
This fact is already proved in Stoll's work.
\end{proof}


Then, since $\zeta$ doesn't vanish on the 1-line, there is a $\sigma<1$ (depending on $T$), so that
the box $[\sigma,1] \times_{ℂ} [-T,T]$ is free of zeros of $\zeta$.
\begin{lemma}[ZetaNoZerosInBox]\label{ZetaNoZerosInBox}\lean{ZetaNoZerosInBox}\leanok
For any $T>0$, there is a constant $\sigma<1$ so that
$$
\zeta(\sigma'+it) \ne 0
$$
for all $|t| \leq T$ and $\sigma' \ge \sigma$.
\end{lemma}


\begin{proof}
\uses{ZetaNoZerosOn1Line}\leanok
Assume not. Then there is a sequence $|t_n| \le T$ and $\sigma_n \to 1$ so that
 $\zeta(\sigma_n + it_n) = 0$.
By compactness, there is a subsequence $t_{n_k} \to t_0$ along which $\zeta(\sigma_{n_k} + it_{n_k}) = 0$.
If $t_0\ne0$, use the continuity of $\zeta$ to get that $\zeta(1 + it_0) = 0$; this is a contradiction.
If $t_0=0$, $\zeta$ blows up near $1$, so can't be zero nearby.
\end{proof}


We now prove that there's an absolute constant $\sigma_0$ so that $\zeta'/\zeta$ is holomorphic on a rectangle $[\sigma_2,2] \times_{ℂ} [-3,3] \setminus \{1\}$.
\begin{lemma}[LogDerivZetaHolcSmallT]\label{LogDerivZetaHolcSmallT}\lean{LogDerivZetaHolcSmallT}\leanok
There is a $\sigma_2 < 1$ so that the function
$$
\frac {\zeta'}{\zeta}(s)
$$
is holomorphic on $\{ \sigma_2 \le \Re s \le 2, |\Im s| \le 3 \} \setminus \{1\}$.
\end{lemma}


\begin{proof}\uses{ZetaNoZerosInBox}\leanok
The derivative of $\zeta$ is holomorphic away from $s=1$; the denominator $\zeta(s)$ is nonzero
in this range by Lemma \ref{ZetaNoZerosInBox}.
\end{proof}


\begin{lemma}[LogDerivZetaHolcLargeT]\label{LogDerivZetaHolcLargeT}\lean{LogDerivZetaHolcLargeT}\leanok
There is an $A>0$ so that for all $T>3$, the function
$
\frac {\zeta'}{\zeta}(s)
$
is holomorphic on $\{1-A/\log^9 T \le \Re s \le 2, |\Im s|\le T \}\setminus\{1\}$.
\end{lemma}


\begin{proof}\uses{ZetaZeroFree}\leanok
The derivative of $\zeta$ is holomorphic away from $s=1$; the denominator $\zeta(s)$ is nonzero
in this range by Lemma \ref{ZetaZeroFree}.
\end{proof}


\begin{lemma}[LogDerivZetaBndUnif]\label{LogDerivZetaBndUnif}\lean{LogDerivZetaBndUnif}\leanok
There exist $A, C > 0$ such that
$$|\frac{\zeta'}{\zeta}(\sigma + it)|\leq C \log |t|^9$$
whenever $|t|>3$ and $\sigma > 1 - A/\log |t|^9$.
\end{lemma}


\begin{proof}\uses{LogDerivZetaBnd}\leanok
For $\sigma$ close to $1$ use Lemma \ref{LogDerivZetaBnd}, otherwise estimate trivially.

