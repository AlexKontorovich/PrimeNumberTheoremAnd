
\section{Summary of results}


In this section we summarize the secondary results known in the literature (or obtained from this project), and (if their proof has already been formalized) provide a proof that reduces them to primary results, as well as implications of primary results to secondary results with appropriate choices of parameters.

Key references:

Dusart: https://piyanit.nl/wp-content/uploads/2020/10/art_10.1007_s11139-016-9839-4.pdf

FKS1: Fiori--Kadiri--Swidninsky arXiv:2204.02588

FKS2: Fiori--Kadiri--Swidninsky arXiv:2206.12557

PT: D. J. Platt and T. S. Trudgian, The error term in the prime number theorem, Math. Comp. 90 (2021), no. 328, 871–881.

JY: D. R. Johnston, A. Yang, Some explicit estimates for the error term in the prime number theorem, arXiv:2204.01980.


\begin{theorem}[PT Corollary 2]\label{thm:pt_2}\lean{PT_Cor2}\leanok
One has
\[
|\pi(x) - \mathrm{Li}(x)| \leq 235 x (\log x)^{0.52} \exp(-0.8 \sqrt{\log x})
\]
for all $x \geq \exp(2000)$.
\end{theorem}


\begin{theorem}[JY Corollary 1.3]\label{thm:jy_13}\lean{JY_Cor13}\leanok
One has
\[
|\pi(x) - \mathrm{Li}(x)| \leq 9.59 x (\log x)^{0.515} \exp(-0.8274 \sqrt{\log x})
\]
for all $x \geq 2$.
\end{theorem}


\begin{theorem}[JY Theorem 1.4]\label{thm:jy_14}\lean{JY_Thm14}\leanok
One has
\[
|\pi(x) - \mathrm{Li}(x)| \leq 0.028 x (\log x)^{0.801} \exp(-0.1853 \log^{3/5} x / (\log \log x)^{1/5}))
\]
for all $x \geq 2$.
\end{theorem}

 TODO: input other results from JY 
 A bound of Staple: The Combinatorial Algorithm for Computing π(x) Ph.D. Thesis, Dalhousie (2015) http://hdl.handle.net/10222/60524

\begin{lemma}[$\pi(10^{15})$]\label{pi_10_15} $\pi(10^{15}) = 29844570422669$
\end{lemma}

 A bound of P. Dusart, Estimates of ψ and θ for large values of without the Riemann hypothesis, Math. Comp. 85 (2016), no. 298, 875–888 :

\begin{lemma}[$\theta(10^{15})$]\label{theta_10_15} $\theta(10^{15}) = 999999965752660.939840$
\end{lemma}


\begin{lemma}[$\mathrm{Li}(10^{15})$]\label{li_10_15} $\mathrm{Li}(10^{15}) = 29844571475286.535901$
\end{lemma}


\begin{lemma}[$E_\pi(10^{15}) - E_\theta(10^{15})$]\label{epi_etheta_10_15} $E_\pi(10^{15}) - E_\theta(10^{15}) = (−2.1087826...) \times 10^{−9}$
\end{lemma}


\begin{remark}[Remark 4, FKS2] one has $\pi(x_0)$ − Li(x_0))/(x_0/\log(x_0)) − \theta(x_0)−x_0 =0$ for $x_0 = 40.787732519...$.
\end{remark}

 Table 1 of D. R. Johnston, A. Yang, Some explicit estimates for the error term in the prime number theorem, arXiv:2204.01980. TODO: expand this

\begin{lemma}[Table 1, JY]\label{jy_table_1} $A_\theta =23.14$, $B = 1.503$, $C = 2.0429...$, $x_0 = e^{10^5}$ are admissible asymptotic bounds for $\theta$.
\end{lemma}

 Table 2 of FKS2. TODO: Expand this
\begin{lemma}[Table 2, FKS2]\label{fks_table_2} If $x_0 = \exp(10^5)$ then $|E_\theta x_0| ≤ 7.7824 \times 10^{−109}$.
\end{lemma}

 Table 4 of FKS2. TODO: Expand this
\begin{lemma}[Table 4, FKS2]\label{fks_table_4} If $x_0 = \exp(10^5)$ then $|E_\pi x_0| ≤ 7.7825 \times 10^{−10^9}$.
\end{lemma}


\begin{lemma}[Combining Table 2 and Table 4 of FKS2]\label{fks_table_24_cor} If $x_0 = \exp(10^5)$ then $|E_\pi x_0 - E_\theta x_0 | ≤ 1.56849 \times 10^{−108}$.
\end{lemma}


\begin{lemma}[Computing $\mu_{asymp}$]\label{fks_remark4_mu} With $A_\theta =23.14$, $B = 1.503$, $C = 2.0429...$, $R = 1$, $x_0 = \exp(10^5)$, and $x_1 = \exp(10^5)$ one has $\mu_{asymp}(x_0,x_1) ≤ 2252.31$.
\end{lemma}


\begin{lemma}[Example of $E_\pi$ bound]\label{fks_remark4_epi} One has an admissible asymptotic bound for $E_\pi$ with parameters $52141.6, 1.503, 2.0429..., 1, \exp(10^5)$.
\end{lemma}


\begin{lemma}[Alternate computation of $\mu_{asymp}$]\label{fks_remark4_mu_alt} With $A_\theta =23.14$, $B = 1.503$, $C = 2.0429...$, $R = 1$, $x_0 = \exp(10^5)$, and $x_1 = \exp(100016)$ one has $\mu_{asymp}(x_0,x_1) ≤ 2.66336 \times 10^{-4}$.
\end{lemma}


\begin{lemma}[Alternate $E_\pi$ bound]\label{fks_remark4_epi_alt} One has an admissible asymptotic bound for $E_\pi$ with parameters $23.15, 1.503, 2.0429..., 1, \exp(100016)$.
\end{lemma}


\begin{theorem}[Dusart Proposition 5.4]\label{thm:Dusart}
There exists a constant \(X_0\) (one may take \(X_0 = 89693\)) with the following property:
for every real \(x \ge X_0\), there exists a prime \(p\) with
\[
  x < p \le x\Bigl(1 + \frac{1}{\log^3 x}\Bigr).
\]
\end{theorem}

 TODO: input other results from Dusart 
