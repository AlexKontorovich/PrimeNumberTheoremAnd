\documentclass{report}

\usepackage{amsmath, amsthm}
\usepackage[showmore, dep_graph, coverage, project=../../]{blueprint}

\theoremstyle{definition}
\newtheorem{definition}{Definition}
\newtheorem{theorem}{Theorem}
\newtheorem{proposition}{Proposition}
\newtheorem{lemma}{Lemma}
\newtheorem{corollary}{Corollary}

\dochome{https://github.com/AlexKontorovich/PrimeNumberTheoremAnd}

\title{Complex Analysis: Has Primitives}
\author{Rutgers Lean Seminar}

%
% This is a blueprint for the HasPrimitives project, written by the Rutgers Lean Group
% 
% This blueprint can be edited like a standard LaTeX document.
%
% There are 3 useful commands that are specific to lean:
%   * \lean{leanref}: is used in a definition, lemma or theorem, and is the
%		      name of the lean object corresponding to the definition,
%		      lemma or theorem.
%   * \leanok: is used in a lemma or theorem: marks that the lemma or theorem
%              has been fully formalized
%   * \uses{ref}: is used in a proof environment: specifies which definitions,
%                 lemma, theorems are used in the proof. Note that 'ref' is a
%                 LaTeX ref, not a lean one.
%

\begin{document}
\maketitle

This project aims to formalize a proof that holomorphic functions on simply connected open sets have primitives.

\chapter{The project}

\section{Main result}

\input{PrimitivesOfSimplyConnected.tex}

\section{Primitives on a Disc}



In this file, we prove the Prime Number Theorem. Continuations of this project aim to extend
this work to primes in progressions (Dirichlet's theorem), Chebytarev's density theorem, etc
etc.




A function is Meromorphic on a rectangle with corners $z$ and $w$ if it is holomorphic off a
(finite) set of poles, none of which are on the boundary of the rectangle.



Discuss polar behavior of meromorphic functions

A




We show that if a function is meromorphic on a rectangle, then the rectangle integral of the
function is equal to the sum of the residues of the function at its poles.



MellinTransform

Mellin Inversion (Goldfeld-Kontorovich)

ChebyshevPsi

ZeroFreeRegion

Hadamard Factorization

Hoffstein-Lockhart + Goldfeld-Hoffstein-Liemann

LSeries (NatPos->C)

RiemannZetaFunction

RectangleIntegral

ResidueTheoremOnRectangle

ArgumentPrincipleOnRectangle

Break rectangle into lots of little rectangles where f is holomorphic, and squares with center at a pole

HasPoleAt f z : TendsTo 1/f (N 0)

Equiv: TendsTo f atTop

Then locally f looks like (z-z_0)^N g

For all c sufficiently small, integral over big rectangle with finitely many poles is equal to rectangle integral localized at each pole.
Rectangles tile rectangles! (But not circles -> circles) No need for toy contours!




\begin{definition}
The Chebyshev Psi function is defined as
$$
\psi(x) = \sum_{n \leq x} \Lambda(n),
$$
where $\Lambda(n)$ is the von Mangoldt function.
\end{definition}




Main Theorem: The Prime Number Theorem
\begin{theorem}[PrimeNumberTheorem]
$$
ψ (x) = x + O(x e^{-c \sqrt{\log x}})
$$
as $x\to \infty$.
\end{theorem}




\end{document}
