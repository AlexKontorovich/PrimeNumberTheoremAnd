\documentclass{report}

\usepackage{amsmath, amsthm}
\usepackage[showmore, dep_graph, coverage, project=../../]{blueprint}

\theoremstyle{definition}
\newtheorem{definition}{Definition}
\newtheorem{theorem}{Theorem}
\newtheorem{proposition}{Proposition}
\newtheorem{lemma}{Lemma}
\newtheorem{corollary}{Corollary}

\dochome{https://AlexKontorovich.github.io/PrimeNumberTheoremAnd/docs}

\title{Prime Number Theorem And ...}

\newcommand{\R}{\mathbb{R}}
\newcommand{\Q}{\mathbb{Q}}
\newcommand{\C}{\mathbb{C}}
\newcommand{\Z}{\mathbb{Z}}
\newcommand{\N}{\mathbb{N}}


\begin{document}
\maketitle

\chapter{The project}

The project github page is https://github.com/AlexKontorovich/PrimeNumberTheoremAnd.

The first main goal is to prove the Prime Number Theorem. Continuations of this project aim to extend
this work to primes in progressions (Dirichlet's theorem), Chebotarev's density theorem, etc
etc.

There are (at least) three approaches to PNT that we may want to pursue simultaneously. The quickest, at this stage, is likely to
follow
 the ``Euclidean Products'' project by Michael Stoll, which has a proof of PNT missing only the Wiener-Ikehara Tauberian theorem.

The second develops some complex analysis (residue calculus on rectangles, argument principle, Mellin transforms), to pull contours and derive a PNT with just an asymptotic formula.

The third approach, which will be the most general of the three, is to: (1) develop the residue calculus et al, as above, (2) add the Hadamard factorization theorem, (3) use it to prove the zero-free region for zeta via Hoffstein-Lockhart+Goldfeld-Hoffstein-Liemann (which generalizes to higher degree L-functions), and (4) use this to prove the prime number theorem with exp-root-log savings.

\chapter{First approach: Wiener-Ikehara Tauberian theorem.}

\section{A Fourier-analytic proof of the Wiener-Ikehara theorem}

The Fourier transform of an absolutely integrable function $\psi: \R \to \C$ is defined by the formula
$$ \hat \psi(u) := \int_\R e(-tu) \psi(t)\ dt$$
where $e(\theta) := e^{2\pi i \theta}$.

Let $f: \N \to \C$ be an arithmetic function such that $\sum_{n=1}^\infty \frac{|f(n)|}{n^\sigma} < \infty$ for all $\sigma>1$.  Then the Dirichlet series
$$ F(s) := \sum_{n=1}^\infty \frac{f(n)}{n^s}$$
is absolutely convergent for $\sigma>1$.


\begin{lemma}[first_fourier]\label{first_fourier}\lean{first_fourier}\leanok  If $\psi: \R \to \C$ is integrable and $x > 0$, then for any $\sigma>1$
  $$ \sum_{n=1}^\infty \frac{f(n)}{n^\sigma} \hat \psi( \frac{1}{2\pi} \log \frac{n}{x} ) = \int_\R F(\sigma + it) \psi(t) x^{it}\ dt.$$
\end{lemma}


\begin{proof}\leanok  By the definition of the Fourier transform, the left-hand side expands as
$$ \sum_{n=1}^\infty \int_\R \frac{f(n)}{n^\sigma} \psi(t) e( - \frac{1}{2\pi} t \log \frac{n}{x})\ dt$$
while the right-hand side expands as
$$ \int_\R \sum_{n=1}^\infty \frac{f(n)}{n^{\sigma+it}} \psi(t) x^{it}\ dt.$$
Since
$$\frac{f(n)}{n^\sigma} \psi(t) e( - \frac{1}{2\pi} t \log \frac{n}{x}) = \frac{f(n)}{n^{\sigma+it}} \psi(t) x^{it}$$
the claim then follows from Fubini's theorem.
\end{proof}


\begin{lemma}[second_fourier]\label{second_fourier}\lean{second_fourier}\leanok If $\psi: \R \to \C$ is continuous and compactly supported and $x > 0$, then for any $\sigma>1$
$$ \int_{-\log x}^\infty e^{-u(\sigma-1)} \hat \psi(\frac{u}{2\pi})\ du = x^{\sigma - 1} \int_\R \frac{1}{\sigma+it-1} \psi(t) x^{it}\ dt.$$
\end{lemma}


\begin{proof}\leanok
The left-hand side expands as
$$ \int_{-\log x}^\infty \int_\R e^{-u(\sigma-1)} \psi(t) e(-\frac{tu}{2\pi})\ dt\ du \atop{?}=
x^{\sigma - 1} \int_\R \frac{1}{\sigma+it-1} \psi(t) x^{it}\ dt$$
so by Fubini's theorem it suffices to verify the identity
\begin{align*}
\int_{-\log x}^\infty e^{-u(\sigma-1)} e(-\frac{tu}{2\pi})\ du
&= \int_{-\log x}^\infty e^{(it - \sigma + 1)u}\ du \\
&= \frac{1}{it - \sigma + 1} e^{(it - \sigma + 1)u}\ \Big|_{-\log x}^\infty \\
&= x^{\sigma - 1} \frac{1}{\sigma+it-1} x^{it}
\end{align*}
\end{proof}


Now let $A \in \C$, and suppose that there is a continuous function $G(s)$ defined on $\mathrm{Re} s \geq 1$ such that $G(s) = F(s) - \frac{A}{s-1}$ whenever $\mathrm{Re} s > 1$.  We also make the Chebyshev-type hypothesis
\begin{equation}\label{cheby}
\sum_{n \leq x} |f(n)| \ll x
\end{equation}
for all $x \geq 1$ (this hypothesis is not strictly necessary, but simplifies the arguments and can be obtained fairly easily in applications).


\begin{lemma}[Preliminary decay bound I]\label{prelim-decay}
If $\psi:\R \to \C$ is absolutely integrable then
$$ |\hat \psi(u)| \leq \| \psi \|_1 $$
for all $u \in \R$. where $C$ is an absolute constant.
\end{lemma}


\begin{proof} Immediate from the triangle inequality.
\end{proof}


\begin{lemma}[Preliminary decay bound II]\label{prelim-decay-2}
If $\psi:\R \to \C$ is absolutely integrable and of bounded variation, and $\psi'$ is bounded variation, then
$$ |\hat \psi(u)| \leq \| \psi \|_{TV} / 2\pi |u| $$
for all non-zero $u \in \R$.
\end{lemma}


\begin{proof} By integration by parts we will have
$$ 2\pi i u \hat \psi(u) = \int _\R e(-tu) \psi'(t)\ dt$$
and the claim then follows from the triangle inequality.
\end{proof}


\begin{lemma}[Preliminary decay bound III]\label{prelim-decay-3}
If $\psi:\R \to \C$ is absolutely integrable, absolutely continuous, and $\psi'$ is of bounded variation, then
$$ |\hat \psi(u)| \leq \| \psi' \|_{TV} / (2\pi |u|)^2$$
for all non-zero $u \in \R$.
\end{lemma}


\begin{proof}\uses{prelim-decay-2} Should follow from previous lemma.
\end{proof}


\begin{lemma}[Decay bound, alternate form]\label{decay-alt}  If $\psi:\R \to \C$ is absolutely integrable, absolutely continuous, and $\psi'$ is of bounded variation, then
$$ |\hat \psi(u)| \leq ( \|\psi\|_1 + \| \psi' \|_{TV} / (2\pi)^2) / (1+|u|^2)$$
for all $u \in \R$.
\end{lemma}


\begin{proof}\uses{prelim-decay, prelim-decay-3, decay} Should follow from previous lemmas.
\end{proof}



It should be possible to refactor the lemma below to follow from Lemma \ref{decay-alt} instead.

\begin{lemma}[Decay bounds]\label{decay}\lean{decay_bounds}\leanok  If $\psi:\R \to \C$ is $C^2$ and obeys the bounds
  $$ |\psi(t)|, |\psi''(t)| \leq A / (1 + |t|^2)$$
  for all $t \in \R$, then
$$ |\hat \psi(u)| \leq C A / (1+|u|^2)$$
for all $u \in \R$, where $C$ is an absolute constant.
\end{lemma}


\begin{proof}\leanok From two integration by parts we obtain the identity
$$ (1+u^2) \hat \psi(u) = \int_{\bf R} (\psi(t) - \frac{u}{4\pi^2} \psi''(t)) e(-tu)\ dt.$$
Now apply the triangle inequality and the identity $\int_{\bf R} \frac{dt}{1+t^2}\ dt = \pi$ to obtain the claim with $C = \pi + 1 / 4 \pi$.
\end{proof}


\begin{lemma}[Limiting Fourier identity]\label{limiting}\lean{limiting_fourier}\leanok  If $\psi: \R \to \C$ is $C^2$ and compactly supported and $x \geq 1$, then
$$ \sum_{n=1}^\infty \frac{f(n)}{n} \hat \psi( \frac{1}{2\pi} \log \frac{n}{x} ) - A \int_{-\log x}^\infty \hat \psi(\frac{u}{2\pi})\ du =  \int_\R G(1+it) \psi(t) x^{it}\ dt.$$
\end{lemma}


\begin{proof}
\uses{first_fourier, second_fourier, decay} \leanok
 By Lemma \ref{first_fourier} and Lemma \ref{second_fourier}, we know that for any $\sigma>1$, we have
  $$ \sum_{n=1}^\infty \frac{f(n)}{n^\sigma} \hat \psi( \frac{1}{2\pi} \log \frac{n}{x} ) - A x^{1-\sigma} \int_{-\log x}^\infty e^{-u(\sigma-1)} \hat \psi(\frac{u}{2\pi})\ du =  \int_\R G(\sigma+it) \psi(t) x^{it}\ dt.$$
  Now take limits as $\sigma \to 1$ using dominated convergence together with \eqref{cheby} and Lemma \ref{decay} to obtain the result.
\end{proof}


\begin{corollary}[Corollary of limiting identity]\label{limiting-cor}\lean{limiting_cor}\leanok  With the hypotheses as above, we have
  $$ \sum_{n=1}^\infty \frac{f(n)}{n} \hat \psi( \frac{1}{2\pi} \log \frac{n}{x} ) = A \int_{-\infty}^\infty \hat \psi(\frac{u}{2\pi})\ du + o(1)$$
  as $x \to \infty$.
\end{corollary}


\begin{proof}
\uses{limiting} \leanok
 Immediate from the Riemann-Lebesgue lemma, and also noting that $\int_{-\infty}^{-\log x} \hat \psi(\frac{u}{2\pi})\ du = o(1)$.
\end{proof}


\begin{lemma}[Smooth Urysohn lemma]\label{smooth-ury}\lean{smooth_urysohn}\leanok  If $I$ is a closed interval contained in an open interval $J$, then there exists a smooth function $\Psi: \R \to \R$ with $1_I \leq \Psi \leq 1_J$.
\end{lemma}


\begin{proof}  \leanok
A standard analysis lemma, which can be proven by convolving $1_K$ with a smooth approximation to the identity for some interval $K$ between $I$ and $J$. Note that we have ``SmoothBumpFunction''s on smooth manifolds in Mathlib, so this shouldn't be too hard...
\end{proof}


\begin{lemma}[Limiting identity for Schwartz functions]\label{schwarz-id}\lean{limiting_cor_schwartz}\leanok  The previous corollary also holds for functions $\psi$ that are assumed to be in the Schwartz class, as opposed to being $C^2$ and compactly supported.
\end{lemma}


\begin{proof}
\uses{limiting-cor, smooth-ury}\leanok
For any $R>1$, one can use a smooth cutoff function (provided by Lemma \ref{smooth-ury} to write $\psi = \psi_{\leq R} + \psi_{>R}$, where $\psi_{\leq R}$ is $C^2$ (in fact smooth) and compactly supported (on $[-R,R]$), and $\psi_{>R}$ obeys bounds of the form
$$ |\psi_{>R}(t)|, |\psi''_{>R}(t)| \ll R^{-1} / (1 + |t|^2) $$
where the implied constants depend on $\psi$.  By Lemma \ref{decay} we then have
$$ \hat \psi_{>R}(u) \ll R^{-1} / (1+|u|^2).$$
Using this and \eqref{cheby} one can show that
$$ \sum_{n=1}^\infty \frac{f(n)}{n} \hat \psi_{>R}( \frac{1}{2\pi} \log \frac{n}{x} ), A \int_{-\infty}^\infty \hat \psi_{>R} (\frac{u}{2\pi})\ du \ll R^{-1} $$
(with implied constants also depending on $A$), while from Lemma \ref{limiting-cor} one has
$$ \sum_{n=1}^\infty \frac{f(n)}{n} \hat \psi_{\leq R}( \frac{1}{2\pi} \log \frac{n}{x} ) = A \int_{-\infty}^\infty \hat \psi_{\leq R} (\frac{u}{2\pi})\ du + o(1).$$
Combining the two estimates and letting $R$ be large, we obtain the claim.
\end{proof}


\begin{lemma}[Bijectivity of Fourier transform]\label{bij}\lean{fourier_surjection_on_schwartz}\leanok  The Fourier transform is a bijection on the Schwartz class. [Note: only surjectivity is actually used.]
\end{lemma}


\begin{proof}
  \leanok
 This is a standard result in Fourier analysis.
It can be proved here by appealing to Mellin inversion, Theorem \ref{MellinInversion}.
In particular, given $f$ in the Schwartz class, let $F : \R_+ \to \C : x \mapsto f(\log x)$ be a function in the ``Mellin space''; then the Mellin transform of $F$ on the imaginary axis $s=it$ is the Fourier transform of $f$.  The Mellin inversion theorem gives Fourier inversion.
\end{proof}


\begin{corollary}[Smoothed Wiener-Ikehara]\label{WienerIkeharaSmooth}\lean{wiener_ikehara_smooth}\leanok
  If $\Psi: (0,\infty) \to \C$ is smooth and compactly supported away from the origin, then,
$$ \sum_{n=1}^\infty f(n) \Psi( \frac{n}{x} ) = A x \int_0^\infty \Psi(y)\ dy + o(x)$$
as $x \to \infty$.
\end{corollary}


\begin{proof}
\uses{bij,schwarz-id}\leanok
 By Lemma \ref{bij}, we can write
$$ y \Psi(y) = \hat \psi( \frac{1}{2\pi} \log y )$$
for all $y>0$ and some Schwartz function $\psi$.  Making this substitution, the claim is then equivalent after standard manipulations to
$$ \sum_{n=1}^\infty \frac{f(n)}{n} \hat \psi( \frac{1}{2\pi} \log \frac{n}{x} ) = A \int_{-\infty}^\infty \hat \psi(\frac{u}{2\pi})\ du + o(1)$$
and the claim follows from Lemma \ref{schwarz-id}.
\end{proof}


Now we add the hypothesis that $f(n) \geq 0$ for all $n$.

\begin{proposition}[Wiener-Ikehara in an interval]
\label{WienerIkeharaInterval}\lean{WienerIkeharaInterval}\leanok
  For any closed interval $I \subset (0,+\infty)$, we have
  $$ \sum_{n=1}^\infty f(n) 1_I( \frac{n}{x} ) = A x |I|  + o(x).$$
\end{proposition}


\begin{proof}
\uses{smooth-ury, WienerIkeharaSmooth} \leanok
  Use Lemma \ref{smooth-ury} to bound $1_I$ above and below by smooth compactly supported functions whose integral is close to the measure of $|I|$, and use the non-negativity of $f$.
\end{proof}


\begin{corollary}[Wiener-Ikehara theorem]\label{WienerIkehara}\lean{WienerIkeharaTheorem'}\leanok
  We have
$$ \sum_{n\leq x} f(n) = A x + o(x).$$
\end{corollary}


\begin{proof}
\uses{WienerIkeharaInterval} \leanok
  Apply the preceding proposition with $I = [\varepsilon,1]$ and then send $\varepsilon$ to zero (using \eqref{cheby} to control the error).
\end{proof}


\section{Weak PNT}

\begin{theorem}[WeakPNT]\label{WeakPNT}\lean{WeakPNT}\leanok  We have
$$ \sum_{n \leq x} \Lambda(n) = x + o(x).$$
\end{theorem}


\begin{proof}
\uses{WienerIkehara, ChebyshevPsi} \leanok
  Already done by Stoll, assuming Wiener-Ikehara.
\end{proof}


\section{Removing the Chebyshev hypothesis}

In this section we do *not* assume the bound \eqref{cheby}, but instead derive it from the other hypotheses.

\begin{lemma}[limiting_fourier_variant]\label{limiting_fourier_variant}\lean{limiting_fourier_variant}\leanok  If $\psi: \R \to \C$ is $C^2$ and compactly supported with $f$ and $\hat \psi$ non-negative, and $x \geq 1$, then
$$ \sum_{n=1}^\infty \frac{f(n)}{n} \hat \psi( \frac{1}{2\pi} \log \frac{n}{x} ) - A \int_{-\log x}^\infty \hat \psi(\frac{u}{2\pi})\ du =  \int_\R G(1+it) \psi(t) x^{it}\ dt.$$
\end{lemma}


\begin{proof}
\uses{first_fourier, second_fourier, decay}  Repeat the proof of Lemma \ref{limiting_fourier_variant}, but use monotone convergence instead of dominated convergence.  (The proof should be simpler, as one no longer needs to establish domination for the sum.)
\end{proof}


\begin{corollary}[crude_upper_bound]\label{crude_upper_bound}\lean{crude_upper_bound}\leanok  If $\psi: \R \to \C$ is $C^2$ and compactly supported with $f$ and $\hat \psi$ non-negative, then there exists a constant $B$ such that
$$ |\sum_{n=1}^\infty \frac{f(n)}{n} \hat \psi( \frac{1}{2\pi} \log \frac{n}{x} )| \leq B$$
for all $x > 0$.
\end{corollary}


\begin{proof}
\uses{limiting_fourier_variant} For $x \geq 1$, this readily follows from the previous lemma and the triangle inequality. For $x < 1$, only a bounded number of summands can contribute and the claim is trivial.
\end{proof}


\begin{corollary}[auto_cheby]\label{auto_cheby}\lean{auto_cheby}\leanok  One has
$$ \sum_{n \leq x} f(n) = O(x)$$
for all $x \geq 1$.
\end{corollary}


\begin{proof}
\uses{crude_upper_bound} By applying Corollary \ref{crude_upper_bound} for a specific compactly supported function $\psi$, one can obtain a bound of the form
$\sum_{(1-\varepsilon)x < n \leq x} f(n) = O(x)$ for all $x$ and some absolute constant $\varepsilon$ (which can be made explicit).  If $C$ is a sufficiently large constant, the claim $|\sum_{n \leq x} f(n)| \leq Cx$ can now be proven by strong induction on $x$, as the claim for $(1-\varepsilon)x$ implies the claim for $x$ by the triangle inequality (and the claim is trivial for $x < 1$).


\begin{corollary}[WienerIkeharaTheorem'']\label{WienerIkeharaTheorem''}\lean{WienerIkeharaTheorem''}\leanok
  We have
$$ \sum_{n\leq x} f(n) = A x + o(x).$$
\end{corollary}


\begin{proof}
\uses{auto_cheby, WienerIkehara}\leanok Use Corollary \ref{auto_cheby} to remove the Chebyshev hypothesis in Theorem \ref{WienerIkehara}.
\end{proof}


\section{The prime number theorem in arithmetic progressions}

\begin{lemma}[WeakPNT_character]\label{WeakPNT_character}\lean{WeakPNT_character}\leanok  If $q ≥ 1$ and $a$ is coprime to $q$, and $\mathrm{Re} s > 1$, we have
$$
\sum_{n: n = a\ (q)} \frac{\Lambda(n)}{n^s} = - \frac{1}{\varphi(q)} \sum_{\chi\ (q)} \overline{\chi(a)} \frac{L'(s,\chi)}{L(s,\chi)}.$$
\end{lemma}


\begin{proof}\leanok  From the Fourier inversion formula on the multiplicative group $(\Z/q\Z)^\times$, we have
$$ 1_{n=a\ (q)} = \frac{\varphi(q)}{q} \sum_{\chi\ (q)} \overline{\chi(a)} \chi(n).$$
On the other hand, from standard facts about L-series we have for each character $\chi$ that
$$
\sum_{n} \frac{\Lambda(n) \chi(n)}{n^s} = - \frac{L'(s,\chi)}{L(s,\chi)}.$$
Combining these two facts, we obtain the claim.
\end{proof}


\begin{proposition}[WeakPNT_AP_prelim]\label{WeakPNT_AP_prelim}\lean{WeakPNT_AP_prelim}\leanok  If $q ≥ 1$ and $a$ is coprime to $q$, the Dirichlet series $\sum_{n \leq x: n = a\ (q)} {\Lambda(n)}{n^s}$ converges for $\mathrm{Re}(s) > 1$ to $\frac{1}{\varphi(q)} \frac{1}{s-1} + G(s)$ where $G$ has a continuous extension to $\mathrm{Re}(s)=1$.
\end{proposition}



\begin{proof}
\uses{ChebyshevPsi, WeakPNT_character}
We expand out the left-hand side using Lemma \ref{WeakPNT_character}.  The contribution of the non-principal characters $\chi$ extend continuously to $\mathrm{Re}(s) = 1$ thanks to the non-vanishing of $L(s,\chi)$ on this line (which should follow from another component of this project), so it suffices to show that for the principal character $\chi_0$, that
$$ -\frac{L'(s,\chi_0)}{L(s,\chi_0)} - \frac{1}{s-1}$$
also extends continuously here.  But we already know that
$$ -\frac{\zeta'(s)}{\zeta(s)} - \frac{1}{s-1}$$
extends, and from Euler product machinery one has the identity
$$ \frac{L'(s,\chi_0)}{L(s,\chi_0)}
= \frac{\zeta'(s)}{\zeta(s)} + \sum_{p|q} \frac{\log p}{p^s-1}.$$
Since there are only finitely many primes dividing $q$, and each summand $\frac{\log p}{p^s-1}$ extends continuously, the claim follows.
\end{proof}


\begin{theorem}[WeakPNT_AP]\label{WeakPNT_AP}\lean{WeakPNT_AP}\leanok  If $q ≥ 1$ and $a$ is coprime to $q$, we have
$$ \sum_{n \leq x: n = a\ (q)} \Lambda(n) = \frac{x}{\varphi(q)} + o(x).$$
\end{theorem}


\begin{proof}\uses{WienerIkehara, WeakPNT_AP_prelim}
Apply Theorem \ref{WienerIkehara} (or Theorem \ref{WienerIkeharaTheorem''}) to Proposition \ref{WeakPNT_AP_prelim}.  (The Chebyshev bound follows from the corresponding bound for $\Lambda$.)
\end{proof}



\section{The Chebotarev density theorem: the case of cyclotomic extensions}

In this section, $K$ is a number field, $L = K(\mu_m)$ for some natural number $m$, and $G = Gal(K/L)$.

The goal here is to prove the Chebotarev density theorem for the case of cyclotomic extensions.


\begin{lemma}[Dedekind_factor]\label{Dedekind_factor}  We have
$$ \zeta_L(s) = \prod_{\chi} L(\chi,s)$$
for $\Re(s) > 1$, where $\chi$ runs over homomorphisms from $G$ to $\C^\times$ and $L$ is the Artin $L$-function.
\end{lemma}



\begin{proof} See Propositions 7.1.16, 7.1.19 of https://www.math.ucla.edu/~sharifi/algnum.pdf .
\end{proof}


\begin{lemma}[Simple pole]\label{Dedekind_pole}  $\zeta_L$ has a simple pole at $s=1$.
\end{lemma}


\begin{proof} See Theorem 7.1.12 of https://www.math.ucla.edu/~sharifi/algnum.pdf .
\end{proof}


\begin{lemma}[Dedekind_nonvanishing]\label{Dedekind_nonvanishing}  For any non-principal character $\chi$ of $Gal(K/L)$, $L(\chi,s)$ does not vanish for $\Re(s)=1$.
\end{lemma}



\begin{proof}\uses{Dedekind_factor, Dedekind_pole} For $s=1$, this will follow from Lemmas \ref{Dedekind_factor}, \ref{Dedekind_pole}. For the rest of the line, one should be able to adapt the arguments for the Dirichet L-function.
\end{proof}


\section{The Chebotarev density theorem: the case of abelian extensions}

(Use the arguments in Theorem 7.2.2 of https://www.math.ucla.edu/~sharifi/algnum.pdf to extend the previous results to abelian extensions (actually just cyclic extensions would suffice))



\section{The Chebotarev density theorem: the general case}

(Use the arguments in Theorem 7.2.2 of https://www.math.ucla.edu/~sharifi/algnum.pdf to extend the previous results to arbitrary extensions



\begin{lemma}[PNT for one character]\label{Dedekind-PNT}  For any non-principal character $\chi$ of $Gal(K/L)$,
$$ \sum_{N \mathfrak{p} \leq x} \chi(\mathfrak{p}) \log N \mathfrak{p}  = o(x).$$
\end{lemma}


\begin{proof}\uses{Dedekind_nonvanishing} This should follow from Lemma \ref{Dedekind_nonvanishing} and the arguments for the Dirichlet L-function. (It may be more convenient to work with a von Mangoldt type function instead of $\log N\mathfrak{p}$).
\end{proof}



\chapter{Second approach}

\section{Residue calculus on rectangles}

\begin{definition}[RectangleIntegral]\label{RectangleIntegral}\lean{RectangleIntegral}\leanok
A RectangleIntegral of a function $f$ is one over a rectangle determined by $z$ and $w$ in $\C$.
We will sometimes denote it by $\int_{z}^{w} f$. (There is also a primed version, which is $1/(2\pi i)$ times the original.)
\end{definition}


It is very convenient to define integrals along vertical lines in the complex plane, as follows.
\begin{definition}[VerticalIntegral]\label{VerticalIntegral}\leanok
Let $f$ be a function from $\mathbb{C}$ to $\mathbb{C}$, and let $\sigma$ be a real number. Then we define
$$\int_{(\sigma)}f(s)ds = \int_{\sigma-i\infty}^{\sigma+i\infty}f(s)ds.$$
\end{definition}

 We also have a version with a factor of $1/(2\pi i)$.


\begin{theorem}[existsDifferentiableOn_of_bddAbove]\label{existsDifferentiableOn_of_bddAbove}\lean{existsDifferentiableOn_of_bddAbove}\leanok
If $f$ is differentiable on a set $s$ except at $c\in s$, and $f$ is bounded above on $s\setminus\{c\}$, then there exists a differentiable function $g$ on $s$ such that $f$ and $g$ agree on $s\setminus\{c\}$.
\end{theorem}


\begin{proof}\leanok
This is the Riemann Removable Singularity Theorem, slightly rephrased from what's in Mathlib. (We don't care what the function $g$ is, just that it's holomorphic.)
\end{proof}


\begin{theorem}[HolomorphicOn.vanishesOnRectangle]\label{HolomorphicOn.vanishesOnRectangle}\lean{HolomorphicOn.vanishesOnRectangle}\leanok
If $f$ is holomorphic on a rectangle $z$ and $w$, then the integral of $f$ over the rectangle with corners $z$ and $w$ is $0$.
\end{theorem}


\begin{proof}\leanok
This is in a Mathlib PR.
\end{proof}


The next lemma allows to zoom a big rectangle down to a small square, centered at a pole.

\begin{lemma}[RectanglePullToNhdOfPole]\label{RectanglePullToNhdOfPole}\lean{RectanglePullToNhdOfPole}\leanok
If $f$ is holomorphic on a rectangle $z$ and $w$ except at a point $p$, then the integral of $f$
over the rectangle with corners $z$ and $w$ is the same as the integral of $f$ over a small square
centered at $p$.
\end{lemma}


\begin{proof}\uses{HolomorphicOn.vanishesOnRectangle}\leanok
Chop the big rectangle with two vertical cuts and two horizontal cuts into smaller rectangles,
the middle one being the desired square. The integral over each of the outer rectangles
vanishes, since $f$ is holomorphic there. (The constant $c$ being ``small enough'' here just means
that the inner square is strictly contained in the big rectangle.)

\end{proof}


\begin{lemma}[ResidueTheoremAtOrigin]\label{ResidueTheoremAtOrigin}
\lean{ResidueTheoremAtOrigin}\leanok
The rectangle (square) integral of $f(s) = 1/s$ with corners $-1-i$ and $1+i$ is equal to $2\pi i$.
\end{lemma}


\begin{proof}\leanok
This is a special case of the more general result above.
\end{proof}


\begin{lemma}[ResidueTheoremOnRectangleWithSimplePole]\label{ResidueTheoremOnRectangleWithSimplePole}
\lean{ResidueTheoremOnRectangleWithSimplePole}\leanok
Suppose that $f$ is a holomorphic function on a rectangle, except for a simple pole
at $p$. By the latter, we mean that there is a function $g$ holomorphic on the rectangle such that, $f = g + A/(s-p)$ for some $A\in\C$. Then the integral of $f$ over the
rectangle is $A$.
\end{lemma}


\begin{proof}
\uses{ResidueTheoremAtOrigin, RectanglePullToNhdOfPole, HolomorphicOn.vanishesOnRectangle}
\leanok
Replace $f$ with $g + A/(s-p)$ in the integral.
The integral of $g$ vanishes by Lemma \ref{HolomorphicOn.vanishesOnRectangle}.
 To evaluate the integral of $1/(s-p)$,
pull everything to a square about the origin using Lemma \ref{RectanglePullToNhdOfPole},
and rescale by $c$;
what remains is handled by Lemma \ref{ResidueTheoremAtOrigin}.
\end{proof}



\section{Mellin transforms}

In this section, we define the Mellin transform (already in Mathlib, thanks to David Loeffler), and prove its inversion formula.

\begin{definition}
Let $f$ be a function from $\mathbb{R}_{>0}$ to $\mathbb{C}$. We define the Mellin transform of $f$ to be the function $\mathcal{M}(f)$ from $\mathbb{C}$ to $\mathbb{C}$ defined by
$$\mathcal{M}(f)(s) = \int_0^\infty f(x)x^{s-1}dx.$$
\end{definition}

[Note: My preferred way to think about this is that we are integrating over the multiplicative group $\mathbb{R}_{>0}$, multiplying by a (not necessarily unitary!) character $|\cdot|^s$, and integrating with respect to the invariant Haar measure $dx/x$. This is very useful in the kinds of calculations carried out below.]




It is very convenient to define integrals along vertical lines in the complex plane, as follows.
\begin{definition}\label{VerticalIntegral}
Let $f$ be a function from $\mathbb{C}$ to $\mathbb{C}$, and let $σ$ be a real number. Then we define
$$\int_{(σ)}f(s)ds = \int_{σ-i\infty}^{σ+i\infty}f(s)ds.$$
\end{definition}



We first prove the following ``Perron-type'' formula.
\begin{lemma}\label{PerronFormula}
For $x>0$ and $σ>1$, we have
$$
\int_{(σ)}\frac{x^s}{s(s+1)}ds = \begin{cases}
1-\frac1x & \text{ if }x>1\\
0 & \text{ if } x<1
\end{cases}.
$$
\end{lemma}



\begin{proof}
Pull contours and collect residues. This only involves rectangles, and everything is absolutely convergent.
\end{proof}



\begin{theorem}\label{MellinInversion}
Let $f$ be a nice function from $\mathbb{R}_{>0}$ to $\mathbb{C}$, and let $σ$ be sufficiently large. Then
$$f(x) = \frac{1}{2\pi i}\int_{(σ)}\mathcal{M}(f)(s)x^{-s}ds.$$
\end{theorem}



\begin{proof}
The proof is from [Goldfeld-Kontorovich 2012].
Integrate by parts twice.
$$
\mathcal{M}(f)(s) = \int_0^\infty f(x)x^{s-1}dx = - \int_0^\infty f'(x)x^s\frac{1}{s}dx = \int_0^\infty f''(x)x^{s+1}\frac{1}{s(s+1)}dx.
$$
Assuming $f$ is Schwartz, say, we now have at least quadratic decay in $s$ of the Mellin transform. Inserting this formula into the inversion formula gives:
$$
RHS = \frac{1}{2\pi i}\left(\int_{(σ)}\int_0^\infty f''(t)t^{s+1}\frac{1}{s(s+1)}dt\right) x^{-s}ds
$$
$$
= \int_0^\infty f''(t) t \left( \frac{1}{2\pi i}\int_{(σ)}(t/x)^s\frac{1}{s(s+1)}ds\right) dt.
$$
Apply the Perron formula to the inside:
$$
= \int_x^\infty f''(t) t \left(1-\frac{x}{t}\right)dt
= -\int_x^\infty f'(t) dt
= f(x),
$$
where we integrated by parts (undoing the first partial integration), and finally applied the fundamental theorem of calculus (undoing the second).
\end{proof}



Finally, we need Mellin Convolutions and properties thereof.
\begin{definition}\label{MellinConvolution}
Let $f$ and $g$ be functions from $\mathbb{R}_{>0}$ to $\mathbb{C}$. Then we define the Mellin convolution of $f$ and $g$ to be the function $f\ast g$ from $\mathbb{R}_{>0}$ to $\mathbb{C}$ defined by
$$(f\ast g)(x) = \int_0^\infty f(y)g(x/y)\frac{dy}{y}.$$
\end{definition}



The Mellin transform of a convolution is the product of the Mellin transforms.
\begin{theorem}\label{MellinConvolutionTransform}
Let $f$ and $g$ be functions from $\mathbb{R}_{>0}$ to $\mathbb{C}$. Then
$$\mathcal{M}(f\ast g)(s) = \mathcal{M}(f)(s)\mathcal{M}(g)(s).$$
\end{theorem}



\begin{proof}
This is a straightforward calculation.
\end{proof}



Let $\psi$ be a bumpfunction supported in $[1/2,2]$; that is, $\psi$ is nonnegative, smooth, and has total mass
$$
\int_0^\infty \psi(x)\frac{dx}{x} = 1.
$$
\begin{theorem}\label{SmoothExistence}
Such a bumpfunction exists.
\end{theorem}



The $\psi$ function has Mellin transform $\mathcal{M}(\psi)(s)$ which is entire and decays (at least) like $1/|s|$.
\begin{theorem}\label{MellinOfPsi}
The Mellin transform of $\psi$ is
$$\mathcal{M}(\psi)(s) =  O\left(\frac{1}{|s|}\right),$$
as $|s|\to\infty$.
\end{theorem}

[Of course it decays faster than any power of $|s|$, but it turns out that we will just need one power.]



\begin{proof}
Integrate by parts once.
\end{proof}



We can make a delta spike out of this bumpfunction, as follows.
\begin{definition}\label{DeltaSpike}
Let $\psi$ be a bumpfunction supported in $[1/2,2]$. Then for any $\epsilon>0$, we define the delta spike $\psi_\epsilon$ to be the function from $\mathbb{R}_{>0}$ to $\mathbb{C}$ defined by
$$\psi_\epsilon(x) = \frac{1}{\epsilon}\psi\left(x^{\frac{1}{\epsilon}}\right).$$
\end{definition}

This spike still has mass one:
\begin{lemma}\label{DeltaSpikeMass}
For any $\epsilon>0$, we have
$$\int_0^\infty \psi_\epsilon(x)\frac{dx}{x} = 1.$$
\end{lemma}



\begin{proof}
Substitute $y=x^{1/\epsilon}$, and use the fact that $\psi$ has mass one, and that $dx/x$ is Haar measure.
\end{proof}



The Mellin transform of the delta spike is easy to compute.
\begin{theorem}\label{MellinOfDeltaSpike}
For any $\epsilon>0$, the Mellin transform of $\psi_\epsilon$ is
$$\mathcal{M}(\psi_\epsilon)(s) = \mathcal{M}(\psi)\left(\epsilon s\right).$$
\end{theorem}



\begin{proof}
Substitute $y=x^{1/\epsilon}$, use Haar measure; direct calculation.
\end{proof}



In particular, for $s=1$, we have that the Mellin transform of $\psi_\epsilon$ is $1+O(\epsilon)$.
\begin{corollary}\label{MellinOfDeltaSpikeAt1}
For any $\epsilon>0$, we have
$$\mathcal{M}(\psi_\epsilon)(1) =
\mathcal{M}(\psi)(\epsilon)= 1+O(\epsilon).$$
\end{corollary}



\begin{proof}
This is immediate from the above theorem and the fact that $\mathcal{M}(\psi)(0)=1$.
\end{proof}



For $X>0$, let $1_X$ be the function from $\mathbb{R}_{>0}$ to $\mathbb{C}$ defined by
$$1_X(x) = \begin{cases}
1 & \text{ if }x\leq X\\
0 & \text{ if }x>X
\end{cases}.$$
This has Mellin transform
\begin{theorem}\label{MellinOf1X}
The Mellin transform of $1_X$ is
$$\mathcal{M}(1_X)(s) = \frac{X^s}{s}.$$
\end{theorem}
[Note: for $X=1$, this already exists in mathlib]



What will be essential for us is properties of the smooth version of $1_X$, obtained as the
 Mellin convolution of $1_X$ with $\psi_\epsilon$.
\begin{definition}\label{Smooth1X}
Let $X>0$ and $\epsilon>0$. Then we define the smooth function $1_{X,\epsilon}$ from $\mathbb{R}_{>0}$ to $\mathbb{C}$ by
$$1_{X,\epsilon} = 1_X\ast\psi_\epsilon.$$
\end{definition}



In particular, we have the following
\begin{lemma}\label{Smooth1XProperties}
Fix $X>0$ and $\epsilon>0$. There is an absolute constant $c>0$ so that:

(1) If $x\leq X(1-c\epsilon)$, then
$$1_{X,\epsilon}(x) = 1.$$

And (2):
if $x\geq X(1+c\epsilon)$, then
$$1_{X,\epsilon}(x) = 0.$$
\end{lemma}



\begin{proof}
This is a straightforward calculation, using the fact that $\psi_\epsilon$ is supported in $[1/2^\epsilon,2^\epsilon]$.
\end{proof}



Combining the above, we have the following Main Lemma of this section on the Mellin transform of $1_{X,\epsilon}$.
\begin{lemma}\label{MellinOfSmooth1X}
Fix $X>0$ and $\epsilon>0$. Then the Mellin transform of $1_{X,\epsilon}$ is
$$\mathcal{M}(1_{X,\epsilon})(s) = \frac{X^s}{s}\left(\mathcal{M}(\psi)\left(\epsilon s\right)\right).$$
At $s=1$, we have
$$\mathcal{M}(1_{X,\epsilon})(1) = X(1+O(\epsilon)).$$
\end{lemma}



\section{Second proof of the prime number theorem}



We have established that zeta doesn't vanish on the 1 line, and has a pole at $s=1$ of order 1.
We also have that
$$
-\frac{\zeta'(s)}{\zeta(s)} = \sum_{n=1}^\infty \frac{\Lambda(n)}{n^s}.
$$

The main object of study is the following inverse Mellin transform, which will turn out to be a smoothed Chebyshev function.
\begin{definition}\label{SmoothedChebyshev}
Fix $X>0$, $\epsilon>0$, and a bumpfunction $\psi$ supported in $[1/2,2]$. Then we define the smoothed Chebyshev function $\psi_{X,\epsilon}$ from $\mathbb{R}_{>0}$ to $\mathbb{C}$ by
$$\psi_{X,\epsilon}(x) = \frac{1}{2\pi i}\int_{(2)}\frac{-\zeta'(s)}{\zeta(s)}1_{X,\epsilon}(x)x^{-s}ds.$$




\section{Second Proof of PNT}

The approach here is completely standard. We follow the use of $\mathcal{M}(\widetilde{1_{\epsilon}})$ as in Kontorovich 2015.




It has already been established that zeta doesn't vanish on the 1 line, and has a pole at $s=1$ of order 1.
We also have that
$$
-\frac{\zeta'(s)}{\zeta(s)} = \sum_{n=1}^\infty \frac{\Lambda(n)}{n^s}.
$$

The main object of study is the following inverse Mellin-type transform, which will turn out to be a smoothed Chebyshev function.
\begin{definition}\label{SmoothedChebyshev}
Fix $\epsilon>0$, and a bumpfunction $\psi$ supported in $[1/2,2]$. Then we define the smoothed Chebyshev function $\psi_{\epsilon}$ from $\mathbb{R}_{>0}$ to $\mathbb{C}$ by
$$\psi_{\epsilon}(X) = \frac{1}{2\pi i}\int_{(2)}\frac{-\zeta'(s)}{\zeta(s)}
\mathcal{M}(\widetilde{1_{\epsilon}})(s)
X^{s}ds.$$
\end{definition}



Inserting the Dirichlet series expansion of the log derivative of zeta, we get the following.
\begin{theorem}\label{SmoothedChebyshevDirichlet}
We have that
$$\psi_{\epsilon}(X) = \sum_{n=1}^\infty \Lambda(n)\widetilde{1_{\epsilon}}(n/X).$$
\end{theorem}



\begin{proof}
We have that
$$\psi_{\epsilon}(X) = \frac{1}{2\pi i}\int_{(2)}\sum_{n=1}^\infty \frac{\Lambda(n)}{n^s}
\mathcal{M}(\widetilde{1_{\epsilon}})(s)
X^{s}ds.$$
We have enough decay (thanks to quadratic decay of $\mathcal{M}(\widetilde{1_{\epsilon}})$) to justify the interchange of summation and integration. We then get
$$\psi_{\epsilon}(X) =
\sum_{n=1}^\infty \Lambda(n)\frac{1}{2\pi i}\int_{(2)}
\mathcal{M}(\widetilde{1_{\epsilon}})(s)
(n/X)^{-s}
ds
$$
and apply the Mellin inversion formula (Theorem \ref{MellinInversion}).
\end{proof}



The smoothed Chebyshev function is close to the actual Chebyshev function.
\begin{theorem}\label{SmoothedChebyshevClose}
We have that
$$\psi_{\epsilon}(X) = \psi(X) + O(\epsilon X \log X).$$
\end{theorem}



\begin{proof}
Take the difference. By Lemma \ref{Smooth1Properties}, the sums agree except when $1-c \epsilon \leq n/X \leq 1+c \epsilon$. This is an interval of length $\ll \epsilon X$, and the summands are bounded by $\Lambda(n) \ll \log X$.
\end{proof}



Returning to the definition of $\psi_{\epsilon}$, fix a large $T$ to be chosen later, and pull contours (via rectangles!) to go
from $2$ up to $2+iT$, then over to $1+iT$, and up from there to $1+i\infty$ (and symmetrically in the lower half plane). Call
this path $\gamma$. The
rectangles involved are all where the integrand is holomorphic, so there is no change.
\begin{theorem}\label{SmoothedChebyshevPull1}
We have that
$$\psi_{\epsilon}(X) = \frac{1}{2\pi i}\int_{\gamma}\frac{-\zeta'(s)}{\zeta(s)}
\mathcal{M}(\widetilde{1_{\epsilon}})(s)
X^{s}ds.$$
\end{theorem}



Then, since $\zeta$ doesn't vanish on the 1-line, there is a $\delta$ (depending on $T$), so that the box $[1-\delta,1] \times_{ℂ} [-T,T]$ is free of zeros of $\zeta$.

The rectangle integral with opposite corners $1-\delta - i T$ and $2+iT$ contains a single pole of $-\zeta'/\zeta$ at $s=1$, and the residue is $1$ (from Theorem \ref{ResidueOfLogDerivative}).





\chapter{Third Approach}

\section{Hadamard factorization}


In this file, we prove the Hadamard Factorization theorem for functions of finite order, and prove that the zeta function
is such.




\section{Hoffstein-Lockhart}


In this file, we use the Hoffstein-Lockhart construction to prove a zero-free region for zeta.

ZeroFreeRegion

Hoffstein-Lockhart + Goldfeld-Hoffstein-Liemann





\section{Strong PNT}

\begin{definition}
The Chebyshev Psi function is defined as
$$
\psi(x) = \sum_{n \leq x} \Lambda(n),
$$
where $\Lambda(n)$ is the von Mangoldt function.
\end{definition}




Main Theorem: The Prime Number Theorem in strong form.
\begin{theorem}[PrimeNumberTheorem]
There is a constant $c > 0$ such that
$$
ψ (x) = x + O(x e^{-c \sqrt{\log x}})
$$
as $x\to \infty$.
\end{theorem}




\chapter{Elementary Corollaries}


\begin{lemma}[finsum_range_eq_sum_range]\label{finsum_range_eq_sum_range}\lean{finsum_range_eq_sum_range}\leanok For any arithmetic function $f$ and real number $x$, one has
$$ \sum_{n \leq x} f(n) = \sum_{n \leq ⌊x⌋_+} f(n)$$
and
$$ \sum_{n < x} f(n) = \sum_{n < ⌈x⌉_+} f(n).$$
\end{lemma}


\begin{proof}\leanok Straightforward. \end{proof}


\begin{theorem}[chebyshev_asymptotic]\label{chebyshev_asymptotic}\lean{chebyshev_asymptotic}\leanok  One has
  $$ \sum_{p \leq x} \log p = x + o(x).$$
\end{theorem}


\begin{proof}
\uses{WeakPNT, finsum_range_eq_sum_range}\leanok
From the prime number theorem we already have
$$ \sum_{n \leq x} \Lambda(n) = x + o(x)$$
so it suffices to show that
$$ \sum_{j \geq 2} \sum_{p^j \leq x} \log p = o(x).$$
Only the terms with $j \leq \log x / \log 2$ contribute, and each $j$ contributes at most $\sqrt{x} \log x$ to the sum, so the left-hand side is $O( \sqrt{x} \log^2 x ) = o(x)$ as required.
\end{proof}


\begin{corollary}[primorial_bounds]  \label{primorial_bounds}\lean{primorial_bounds}\leanok
We have
  $$ \prod_{p \leq x} p = \exp( x + o(x) )$$
\end{corollary}


\begin{proof}\leanok
\uses{chebyshev_asymptotic}
  Exponentiate Theorem \ref{chebyshev_asymptotic}.
\end{proof}


\begin{theorem}[pi_asymp]\label{pi_asymp}\lean{pi_asymp}\leanok
There exists a function $c(x)$ such that $c(x) = o(1)$ as $x \to \infty$ and
$$ \pi(x) = (1 + c(x)) \int_2^x \frac{dt}{\log t}$$
for all $x$ large enough.
\end{theorem}


\begin{proof}\leanok
\uses{chebyshev_asymptotic}
We have the identity
$$ \pi(x) = \frac{1}{\log x} \sum_{p \leq x} \log p
+ \int_2^x (\sum_{p \leq t} \log p) \frac{dt}{t \log^2 t}$$
as can be proven by interchanging the sum and integral and using the fundamental theorem of calculus.  For any $\eps$, we know from Theorem \ref{chebyshev_asymptotic} that there is $x_\eps$ such that
$\sum_{p \leq t} \log p = t + O(\eps t)$ for $t \geq x_\eps$, hence for $x \geq x_\eps$
$$ \pi(x) = \frac{1}{\log x} (x + O(\eps x))
+ \int_{x_\eps}^x (t + O(\eps t)) \frac{dt}{t \log^2 t} + O_\eps(1)$$
where the $O_\eps(1)$ term can depend on $x_\eps$ but is independent of $x$.  One can evaluate this after an integration by parts as
$$ \pi(x) = (1+O(\eps)) \int_{x_\eps}^x \frac{dt}{\log t} + O_\eps(1)$$
$$  = (1+O(\eps)) \int_{2}^x \frac{dt}{\log t} $$
for $x$ large enough, giving the claim.
\end{proof}


\begin{corollary}[pi_alt]\label{pi_alt}\lean{pi_alt}\leanok  One has
$$ \pi(x) = (1+o(1)) \frac{x}{\log x}$$
as $x \to \infty$.
\end{corollary}


\begin{proof}\leanok
\uses{pi_asymp}
An integration by parts gives
  $$ \int_2^x \frac{dt}{\log t} = \frac{x}{\log x} - \frac{2}{\log 2} + \int_2^x \frac{dt}{\log^2 t}.$$
We have the crude bounds
$$ \int_2^{\sqrt{x}} \frac{dt}{\log^2 t} = O( \sqrt{x} )$$
and
$$ \int_{\sqrt{x}}^x \frac{dt}{\log^2 t} = O( \frac{x}{\log^2 x} )$$
and combining all this we obtain
$$ \int_2^x \frac{dt}{\log t} = \frac{x}{\log x} + O( \frac{x}{\log^2 x} )$$
$$ = (1+o(1)) \frac{x}{\log x}$$
and the claim then follows from Theorem \ref{pi_asymp}.
\end{proof}


Let $p_n$ denote the $n^{th}$ prime.

\begin{proposition}[pn_asymptotic]\label{pn_asymptotic}\lean{pn_asymptotic}\leanok
 One has
  $$ p_n = (1+o(1)) n \log n$$
as $n \to \infty$.
\end{proposition}


\begin{proof}
\uses{pi_alt}\leanok
Use Corollary \ref{pi_alt} to show that for any $\eps>0$, and for $n$ sufficiently large, the number of primes up to $(1-\eps) n \log n$ is less than $n$, and the number of primes up to $(1+\eps) n \log n$ is greater than $n$.
\end{proof}


\begin{corollary}[pn_pn_plus_one] \label{pn_pn_plus_one}\lean{pn_pn_plus_one}\leanok
We have $p_{n+1} - p_n = o(p_n)$
  as $n \to \infty$.
\end{corollary}


\begin{proof}
\uses{pn_asymptotic}\leanok
  Easy consequence of preceding proposition.
\end{proof}


\begin{corollary}[prime_between]  \label{prime_between}\lean{prime_between}\leanok
For every $\eps>0$, there is a prime between $x$ and $(1+\eps)x$ for all sufficiently large $x$.
\end{corollary}


\begin{proof}
\uses{pi_alt}\leanok
Use Corollary \ref{pi_alt} to show that $\pi((1+\eps)x) - \pi(x)$ goes to infinity as $x \to \infty$.
\end{proof}


\begin{proposition}\label{mun}\lean{sum_mobius_div_self_le}\leanok
We have $|\sum_{n \leq x} \frac{\mu(n)}{n}| \leq 1$.
\end{proposition}


\begin{proof}\leanok
From M\"obius inversion $1_{n=1} = \sum_{d|n} \mu(d)$ and summing we have
  $$ 1 = \sum_{d \leq x} \mu(d) \lfloor \frac{x}{d} \rfloor$$
  for any $x \geq 1$. Since $\lfloor \frac{x}{d} \rfloor = \frac{x}{d} - \epsilon_d$ with
  $0 \leq \epsilon_d < 1$ and $\epsilon_x = 0$, we conclude that
  $$ 1 ≥ x \sum_{d \leq x} \frac{\mu(d)}{d} - (x - 1)$$
  and the claim follows.
\end{proof}


\begin{proposition}[M\"obius form of prime number theorem]\label{mu-pnt}\lean{mu_pnt}\leanok  We have $\sum_{n \leq x} \mu(n) = o(x)$.
\end{proposition}


\begin{proof}
\uses{mun, WeakPNT}
From the Dirichlet convolution identity
  $$ \mu(n) \log n = - \sum_{d|n} \mu(d) \Lambda(n/d)$$
and summing we obtain
$$ \sum_{n \leq x} \mu(n) \log n = - \sum_{d \leq x} \mu(d) \sum_{m \leq x/d} \Lambda(m).$$
For any $\eps>0$, we have from the prime number theorem that
$$ \sum_{m \leq x/d} \Lambda(m) = x/d + O(\eps x/d) + O_\eps(1)$$
(divide into cases depending on whether $x/d$ is large or small compared to $\eps$).
We conclude that
$$ \sum_{n \leq x} \mu(n) \log n = - x \sum_{d \leq x} \frac{\mu(d)}{d} + O(\eps x \log x) + O_\eps(x).$$
Applying \eqref{mun} we conclude that
$$ \sum_{n \leq x} \mu(n) \log n = O(\eps x \log x) + O_\eps(x).$$
and hence
$$ \sum_{n \leq x} \mu(n) \log x = O(\eps x \log x) + O_\eps(x) + O( \sum_{n \leq x} (\log x - \log n) ).$$
From Stirling's formula one has
$$  \sum_{n \leq x} (\log x - \log n) = O(x)$$
thus
$$ \sum_{n \leq x} \mu(n) \log x = O(\eps x \log x) + O_\eps(x)$$
and thus
$$ \sum_{n \leq x} \mu(n) = O(\eps x) + O_\eps(\frac{x}{\log x}).$$
Sending $\eps \to 0$ we obtain the claim.
\end{proof}


\begin{proposition}\label{lambda-pnt}\lean{lambda_pnt}\leanok
We have $\sum_{n \leq x} \lambda(n) = o(x)$.
\end{proposition}


\begin{proof}
\uses{mu-pnt}
From the identity
  $$ \lambda(n) = \sum_{d^2|n} \mu(n/d^2)$$
and summing, we have
$$ \sum_{n \leq x} \lambda(n) = \sum_{d \leq \sqrt{x}} \sum_{n \leq x/d^2} \mu(n).$$
For any $\eps>0$, we have from Proposition \ref{mu-pnt} that
$$ \sum_{n \leq x/d^2} \mu(n) = O(\eps x/d^2) + O_\eps(1)$$
and hence on summing in $d$
$$ \sum_{n \leq x} \lambda(n) = O(\eps x) + O_\eps(x^{1/2}).$$
Sending $\eps \to 0$ we obtain the claim.
\end{proof}



\begin{proposition}[Alternate M\"obius form of prime number theorem]\label{mu-pnt-alt}\lean{mu_pnt_alt}\leanok  We have $\sum_{n \leq x} \mu(n)/n = o(1)$.
\end{proposition}


\begin{proof}
\uses{mu-pnt}
As in the proof of Theorem \ref{mun}, we have
  $$ 1 = \sum_{d \leq x} \mu(d) \lfloor \frac{x}{d} \rfloor$$
  $$ = x \sum_{d \leq x} \frac{\mu(d)}{d} - \sum_{d \leq x} \mu(d) \{ \frac{x}{d} \}$$
so it will suffice to show that
$$ \sum_{d \leq x} \mu(d) \{ \frac{x}{d} \} = o(x).$$
Let $N$  be a natural number.  It suffices to show that
$$ \sum_{d \leq x} \mu(d) \{ \frac{x}{d} \} = O(x/N).$$
if $x$ is large enough depending on $N$.
We can split the left-hand side as the sum of
$$ \sum_{d \leq x/N} \mu(d) \{ \frac{x}{d} \} $$
and
$$ \sum_{j=1}^{N-1} \sum_{x/(j+1) < d \leq x/j} \mu(d) (x/d - j).$$
The first term is clearly $O(x/N)$.  For the second term, we can use Theorem \ref{mu-pnt} and summation by parts (using the fact that $x/d-j$ is monotone and bounded) to find that
$$ \sum_{x/(j+1) < d \leq x/j} \mu(d) (x/d - j) = o(x)$$
for any given $j$, so in particular
$$ \sum_{x/(j+1) < d \leq x/j} \mu(d) (x/d - j) = O(x/N^2)$$
for all $j=1,\dots,N-1$ if $x$ is large enough depending on $N$.  Summing all the bounds, we obtain the claim.
\end{proof}


\section{Consequences of the PNT in arithmetic progressions}

\begin{theorem}[Prime number theorem in AP]\label{chebyshev_asymptotic_pnt}\lean{chebyshev_asymptotic_pnt}\leanok  If $a\ (q)$ is a primitive residue class, then one has
  $$ \sum_{p \leq x: p = a\ (q)} \log p = \frac{x}{\phi(q)} + o(x).$$
\end{theorem}


\begin{proof}
\uses{chebyshev_asymptotic}
This is a routine modification of the proof of Theorem \ref{chebyshev_asymptotic}.
\end{proof}


\begin{corollary}[Dirichlet's theorem]\label{dirichlet_thm}\lean{dirichlet_thm}\leanok  Any primitive residue class contains an infinite number of primes.
\end{corollary}


\begin{proof}
\uses{chebyshev_asymptotic_pnt}
If this were not the case, then the sum $\sum_{p \leq x: p = a\ (q)} \log p$ would be bounded in $x$, contradicting Theorem \ref{chebyshev_asymptotic_pnt}.
\end{proof}
-/

/-%%
\section{Consequences of the Chebotarev density theorem}



\begin{lemma}[Cyclotomic Chebotarev]\label{Chebotarev-cyclic}  For any $a$ coprime to $m$,
$$ \sum_{N \mathfrak{p} \leq x; N \mathfrak{p} = a\ (m)} \log N \mathfrak{p}  =
\frac{1}{|G|} \sum_{N \mathfrak{p} \leq x} \log N \mathfrak{p}.$$
\end{lemma}


\begin{proof}\uses{Dedekind-PNT, WeakPNT_AP} This should follow from Lemma \ref{Dedekind-PNT} by a Fourier expansion.
\end{proof}




\end{document}
