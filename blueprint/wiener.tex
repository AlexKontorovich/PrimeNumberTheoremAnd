\section{A Fourier-analytic proof of the Wiener-Ikehara theorem}

The Fourier transform of an absolutely integrable function $\psi: \R \to \C$ is defined by the formula
$$ \hat \psi(u) := \int_\R e(-tu) \psi(t)\ dt$$
where $e(\theta) := e^{2\pi i \theta}$.

Let $f: \N \to \C$ be an arithmetic function such that $\sum_{n=1}^\infty \frac{|f(n)|}{n^\sigma} < \infty$ for all $\sigma>1$.  Then the Dirichlet series
$$ F(s) := \sum_{n=1}^\infty \frac{f(n)}{n^s}$$
is absolutely convergent for $\sigma>1$.

\begin{lemma}[First Fourier identity]\label{first-fourier}  If $\psi: \R \to \C$ is continuous and compactly supported and $x > 0$, then for any $\sigma>1$
  $$ \sum_{n=1}^\infty \frac{f(n)}{n^\sigma} \hat \psi( \frac{1}{2\pi} \log \frac{n}{x} ) = \int_\R F(\sigma + it) \psi(t) x^{it}\ dt.$$
\end{lemma}

\begin{proof}  By the definition of the Fourier transform, the left-hand side expands as
$$ \sum_{n=1}^\infty \int_\R \frac{f(n)}{n^\sigma} \psi(t) e( - \frac{1}{2\pi} t \log \frac{n}{x})\ dt$$
while the right-hand side expands as
$$ \int_\R \sum_{n=1}^\infty \frac{f(n)}{n^{\sigma+it}} \psi(t) x^{it}\ dt.$$
Since
$$\frac{f(n)}{n^\sigma} \psi(t) e( - \frac{1}{2\pi} t \log \frac{n}{x}) = \frac{f(n)}{n^{\sigma+it}} \psi(t) x^{it}$$
the claim then follows from Fubini's theorem.
\end{proof}

\begin{lemma}[Second Fourier identity]\label{second-fourier} If $\psi: \R \to \C$ is continuous and compactly supported and $x > 0$, then for any $\sigma>1$
$$ \int_{-\log x}^\infty e^{-u(\sigma-1)} \hat \psi(\frac{u}{2\pi})\ du = x^{\sigma - 1} \int_\R \frac{1}{\sigma+it-1} \psi(t) x^{it}\ dt.$$
\end{lemma}

\begin{proof}  The left-hand side expands as
  $$ \int_{-\log x}^\infty \int_\R e^{-u(\sigma-1)} \psi(t) e(-\frac{tu}{2\pi})\ dt du = x^{\sigma - 1} \int_\R \frac{1}{\sigma+it-1} \psi(t) x^{it}\ dt$$
  so by Fubini's theorem it suffices to verify the identity
$$ \int_{-\log x}^\infty \int_\R e^{-u(\sigma-1)} e(-\frac{tu}{2\pi})\ du = x^{\sigma - 1} \frac{1}{\sigma+it-1} x^{it}$$
which is a routine calculation.
\end{proof}

Now let $A \in \C$, and suppose that there is a continuous function $G(s)$ defined on $\mathrm{Re} s \geq 1$ such that $G(s) = F(s) - \frac{A}{s-1}$ whenever $\mathrm{Re} s > 1$.  We also make the Chebyshev-type hypothesis
\begin{equation}\label{cheby}
\sum_{n \leq x} |f(n)| \ll x
\end{equation}
for all $x \geq 1$ (this hypothesis is not strictly necessary, but simplifies the arguments and can be obtained fairly easily in applications).

\begin{lemma}[Decay bounds]\label{decay}  If $\psi:\R \to \C$ is $C^2$ and obeys the bounds
  $$ |\psi(t)|, |\psi''(t)| \leq A / (1 + |t|^2)$$
  for all $t \in \R$, then
$$ |\hat \psi(u)| \leq C A / (1+|u|^2)$$
for all $u \in \R$, where $C$ is an absolute constant.
\end{lemma}

\begin{proof} This follows from a standard integration by parts argument.
\end{proof}

\begin{lemma}[Limiting Fourier identity]\label{limiting}  If $\psi: \R \to \C$ is $C^2$ and compactly supported and $x \geq 1$, then
$$ \sum_{n=1}^\infty \frac{f(n)}{n} \hat \psi( \frac{1}{2\pi} \log \frac{n}{x} ) - A \int_{-\log x}^\infty \hat \psi(\frac{u}{2\pi})\ du =  \int_\R G(1+it) \psi(t) x^{it}\ dt.$$
\end{lemma}

\begin{proof}  By the preceding two lemmas, we know that for any $\sigma>1$, we have
  $$ \sum_{n=1}^\infty \frac{f(n)}{n^\sigma} \hat \psi( \frac{1}{2\pi} \log \frac{n}{x} ) - A x^{1-\sigma} \int_{-\log x}^\infty e^{-u(\sigma-1)} \hat \psi(\frac{u}{2\pi})\ du =  \int_\R G(\sigma+it) \psi(t) x^{it}\ dt.$$
  Now take limits as $\sigma \to 1$ using dominated convergence together with \eqref{cheby} and Lemma \ref{decay} to obtain the result.
\end{proof}

\begin{corollary}\label{limiting-cor}  With the hypotheses as above, we have
  $$ \sum_{n=1}^\infty \frac{f(n)}{n} \hat \psi( \frac{1}{2\pi} \log \frac{n}{x} ) = A \int_{-\infty}^\infty \hat \psi(\frac{u}{2\pi})\ du + o(1)$$
  as $x \to \infty$.
\end{corollary}

\begin{proof} Immediate from the Riemann-Lebesgue lemma, and also noting that $\int_{-\infty}^{-\log x} \hat \psi(\frac{u}{2\pi})\ du = o(1)$.
\end{proof}

\begin{lemma}\label{schwarz-id}  The previous corollary also holds for functions $\psi$ that are assumed to be in the Schwartz class, as opposed to being $C^2$ and compactly supported.
\end{lemma}

\begin{proof}
For any $R>1$, one can use a smooth cutoff function to write $\psi = \psi_{\leq R} + \psi_{>R}$, where $\psi_{\leq R}$ is $C^2$ (in fact smooth) and compactly supported (on $[-R,R]$), and $\psi_{>R}$ obeys bounds of the form
$$ |\psi_{>R}(t)|, |\psi''_{>R}(t)| \ll R^{-1} / (1 + |t|^2) $$
where the implied constants depend on $\psi$.  By Lemma \ref{decay} we then have
$$ \hat \psi_{>R}(u) \ll R^{-1} / (1+|u|^2).$$
Using this and \eqref{cheby} one can show that
$$ \sum_{n=1}^\infty \frac{f(n)}{n} \hat \psi_{>R}( \frac{1}{2\pi} \log \frac{n}{x} ), A \int_{-\infty}^\infty \hat \psi_{>R} (\frac{u}{2\pi})\ du \ll R^{-1} $$
(with implied constants also depending on $A$), while from Lemma \ref{limiting-cor} one has
$$ \sum_{n=1}^\infty \frac{f(n)}{n} \hat \psi_{\leq R}( \frac{1}{2\pi} \log \frac{n}{x} ) = A \int_{-\infty}^\infty \hat \psi_{\leq R} (\frac{u}{2\pi})\ du + o(1).$$
Combining the two estimates and letting $R$ be large, we obtain the claim.
\end{proof}

\begin{lemma}\label{bij}  The Fourier transform is a bijection on the Schwartz class.
\end{lemma}

\begin{proof}  This is a standard result in Fourier analysis.
\end{proof}

\begin{corollary}  If $\Psi: (0,\infty) \to \C$ is smooth and compactly supported away from the origin, then, then
$$ \sum_{n=1}^\infty f(n) \Psi( \frac{n}{x} ) = A x \int_0^\infty \Psi(y)\ dy + o(x)$$
as $u \to \infty$.
\end{corollary}

\begin{proof} By Lemma \ref{bij}, we can write
$$ y \Psi(y) = \hat \psi( \frac{1}{2\pi} \log y )$$
for all $y>0$ and some Schwartz function $\psi$.  Making this substitution, the claim is then equivalent after standard manipulations to
$$ \sum_{n=1}^\infty \frac{f(n)}{n} \hat \psi( \frac{1}{2\pi} \log \frac{n}{x} ) = A \int_{-\infty}^\infty \hat \psi(\frac{u}{2\pi})\ du + o(1)$$
and the claim follows from Lemma \ref{schwarz-id}.
\end{proof}

\begin{lemma}[Smooth Urysohn lemma]\label{smooth-ury}  If $I$ is a closed interval contained in an open interval $J$, then there exists a smooth function $\Psi: \R \to \R$ with $1_I \leq \Psi \leq 1_J$.
\end{lemma}

\begin{proof}  A standard analysis lemma, which can be proven by convolving $1_K$ with a smooth approximation to the identity for some interval $K$ between $I$ and $J$.
\end{proof}

Now we add the hypothesis that $f(n) \geq 0$ for all $n$.

\begin{proposition}  For any closed interval $I \subset (0,+\infty)$, we have
  $$ \sum_{n=1}^\infty f(n) 1_I( \frac{n}{x} ) = A x |I|  + o(x).$$
\end{proposition}

\begin{proof}  Use Lemma \ref{smooth-ury} to bound $1_I$ above and below by smooth compactly supported functions whose integral is close to the measure of $|I|$, and use the non-negativity of $f$.
\end{proof}

\begin{corollary}  We have
$$ \sum_{n\leq x} f(n) = A x |I|  + o(x).$$
\end{corollary}

\begin{proof}
  Apply the preceding proposition with $I = [\varepsilon,1]$ and then send $\varepsilon$ to zero (using \eqref{cheby} to control the error).
\end{proof}
